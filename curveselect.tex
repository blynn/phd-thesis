\chapter{Curve Selection}

TODO: do we need to define embedding degrees for subgroups?

Let $E$ be an elliptic curve defined over a finite field $K$.
Let $P \in E(K)$ be a point of prime order $r$. Let $G = \langle P \rangle$.

If $L$ is a field extension of $K$
that contains the $r$th roots of unity (so $r$ divides $|L| - 1$),
then the Tate pairing
on $G$ can be computed by performing operations in $L$.

On the other hand,
let $E[r]$ denote the group of points of order $r$.
The Weil pairing is defined on $E[r]$. For some field extension
$L'$ of $K$, $E(L')$ contains $E[r]$, and the Weil pairing can be
computed by performing operations in $L'$.

In order for the Tate pairing to be efficiently computable,
operations must be efficient in $L$.
Similarly, for the Weil pairing to be efficiently
computable, operations must be efficient in $L'$.

Thus we seek fields $L$ or $L'$ that are small enough so that field
operations are still fast.

It turns out that such a field $L'$ will always contain the $n$th
roots of unity, but the converse is not true: it is possible
for a field $L$ to contain the $r$th roots of unity but
at the same time part of $E[r]$ lies outside $E(L)$.

However, the converse is almost true:

\section {The Embedding Degree}

\begin{theorem} \cite{bk}
Let $E$ be an elliptic curve defined over $\mathbb{F}_q$.
Let $r$ be a prime dividing $N = \#(\mathbb{F}_q$) with
$r \nmid q-1$.
Then for any positive integer $k$,
$E(\mathbb{F}_{q^k})$ contains all $r^2$
points of order $r$ if and only if $r \mid q^k - 1$.

\end{theorem}

\begin{proof}
It is well-known that if $E(\mathbb{F}_{q^k})$ contains $E[r]$
then $r \mid q^k - 1$, even without assuming $r \mid N$ or
$r \nmid q-1$.

Conversely, suppose $k > 1$ and $r \mid q^k - 1$.
Let $\Phi$ denote the Frobenius map. Consider the subgroup $T$ of $E[r]$
consisting of all points of trace zero, that is
\[
T = \{ Q \in E[r] : \tr Q = Q + \Phi(Q) + ... + \Phi^{k-1}(Q) = O \}
\]
(The group $T$ may be explicitly constructed by using the map $P \mapsto
P - \Phi(P)$ on points of $E(\mathbb{F}_{q^k})$.)
Now we have $\Phi(T) = T$, and also $T$ is not contained
in $E(\mathbb{F}_q)$ by assumption.

Hence $T$ is an eigenspace of $\Phi$, but not the $1$-eigenspace. Since the
eigenvalues of $\Phi$ must be $1$ and $q$, we see that $T$ must be the
$q$-eigenspace of $\Phi$ and hence
\[
\Phi^k(Q) = q^k Q = Q
\]
since $r | q^k - 1$. Thus $T$, like $E(\mathbb{F}_q)$ is fixed under
$\Phi^k$, and since these groups are linearly independent they generate all
of $E[r]$, implying that all of $E[r]$ is fixed under $\Phi^k$. Hence
$E[r] \subset E(\mathbb{F}_{q^k})$.
\end{proof}

\begin{definition}
Let $E$ be an elliptic curve defined over $K = \mathbb{F}_q$.
Let $P \in E(K)$ be a point of order $r$.
Let $G = \langle P \rangle$.
The \emph{embedding degree} of $G$ is the smallest positive integer $k$
such that $r \mid q^k - 1$.

If $k$ is the smallest positive integer such that
$\#E(\Fq) | q^k - 1$
then we say the embedding degree of the curve $E$ is $k$.
\end{definition}

Thus we have shown that if the embedding degree $k$ is greater than 1,
then both the Tate pairing and the Weil pairing may be computed by
performing field arithmetic in $\mathbb{F}_{q^k}$.

However, if the embedding degree $k$ is 1, the Tate pairing may
be computable in $\mathbb{F}_q$, but the Weil pairing may not be.
Usually this is not significant because we shall see we prefer
$k > 1$. However, the Tate pairing turns out to have other benefits.

In practice, we construct a curve $E(\Fq)$ and find the smallest
possible $k$ such that $\#E(\Fq) | q^k - 1$, that is the embedding degree
of the curve is $k$. Then we take $r$ to be a large prime factor of
$\#E(\Fq)$ so that $k$ is also the embedding degree of $E(\Fq)[r]$
with high probability.

For example, consider the curve $E: Y^2 = X^3 + X + 6$ which has
18 points over $\mathbb{F}_{19}$.
The point $R=(0,5)$ generates a cyclic group of order 18,
so the point $P = 6R = (12,13)$ generates a cyclic group of order 3.

We see $E[3]$ cannot be contained in $\langle R \rangle = E(\mathbb{F}_{19})$
since $E[3]$ is not cyclic, thus the Weil pairing cannot be computed.
(It turns out we must move to $\mathbb{F}_{19^3}$ to do so.)

In contrast, $3 \mid 19-1$ hence the Tate pairing can be
computed in $\mathbb{F}_{19}$. In fact, we find $e(P, R + 3G) = 11$, where
$e$ denotes the Tate pairing.

\section {Security Considerations}

Thus our goal is to find curves $E(\mathbb{F}_q)$ containing a subgroup
$G$ of prime order $r$ with embedding degree $k$ small enough so that
operations in $\mathbb{F}_{q^k}$ are efficiently computable.
All other things being equal, we prefer $q$ to be small as possible so
that operations in $G$ are fast as possible and can be represented by
as few bits as possible.

On such curves, the Tate pairing is always efficiently computable.
If $k > 1$ then the Weil pairing is also efficiently computable.

On the other hand, we must consider the lower bounds on $q$ and $q^k$.
Recall elliptic curves are attractive for cryptography
because unlike finite fields,
no specialized discrete logarithm algorithm is known for a general elliptic
curve.

Generally speaking, for a cryptographically useful elliptic curve
we need only ensure that $r$ is large enough to foil generic attacks
(e.g. Pollard rho, baby-step-giant-step).

However, a low embedding degree $k$ complicates matters. A bilinear map
sends points of an elliptic curve to elements of a finite field
$\mathbb{F}_{q^k}$. In particular, given a subgroup $G$ of $E(\mathbb{F}_q)$
of prime order $r$, by bilinearity and nondegeneracy we have
\[ \dlog (P, nP) = \dlog ( e(P,P), e(P, nP) )
= \dlog (e(P,P),e(P,P)^n) = n \]
for $n \in \{0,...,r-1\}$.
Thus if the discrete logarithm problem can be solved
in $\mathbb{F}_{q^k}$, then it can also be solved in $G$ \cite{mov,fr}.

Hence we must also ensure $q^k$ is large enough so that
finite field discrete logarithm algorithms such as index calculus are
infeasible in $\mathbb{F}_{q^k}$.

The curves used in pairing-based cryptography are special in other senses.
They are supersingular, or have complex multiplication. This possibly means
they are vulnerable, but no specific attacks for either case are known.

We have now described all necessary conditions on the sizes of
$r$ and $q^k$:

\begin{enumerate}
\item
$r$ must be a large enough prime so that generic discrete logarithm attacks
in a group of order $r$ are ineffective. Since $q \approx \#E(\mathbb{F}_q)$,
this places a similar lower bound on $q$.
\item
$q$ ought to be as small as possible, so that computations in $\mathbb{F}_q$
are as fast as possible.
\item
$q^k$ must be large enough so that finite field discrete logarithm attacks
in $\mathbb{F}_{q^k}$ are ineffective.
\item
$q^k$ must be small enough so that operations in $\mathbb{F}_{q^k}$
are efficient. $q^k$ should be small as possible so that operations
are as fast as possible.
\end{enumerate}

Of course, the first three statements are true for any cryptographically
useful elliptic curve, not just for pairing-based cryptography.

Currently it is considered acceptable to have $r$ about 160-bits.
As for $q^k$, 1024 bits is adequate for many applications, and calculations
in fields of this size can certainly be performed.
Ideally we have $r \approx \#E(\mathbb{F}_q) \approx q$
and hence $q$ is also about 160 bits. This gives an
embedding degree $k$ around $1024 / 160 = 6.4$.

We shall see how to construct curves of embedding degree 6 and 12.
For the near future, embedding degree 6 should be reasonable if used along
with an $r$ that is at lesat $170$ bits long,
and embedding degree 12 curves may be more desirable as time passes.

\section {Short Signatures}

At this point we can see why the signature scheme introduced
in the first chapter
yields short signatures. Recall a signature is a single group element.
We now know that this group can be instantiated as a subgroup of points
of a certain order on an elliptic curve $E$, thus a signature is simply
a point on $E$ over some field $\F_q$ where $q$ is about 160 bits.

A point has two coordinates, but we can discard the $y$-coordinate entirely,
and represent signatures by their $x$-coordinate only if we modify
verification as follows:
\begin{enumerate}
\item
Given an $x$-coordinate of a signature, find any solution $y$
in the curve equation $E$.
\item
If the signature $(x,y)$ does not verify,
check if the signature $(x, -y)$ verifies.
\item
If the signature still does not verify then reject it.
\end{enumerate}

Thus signatures are roughly 160 bits in length.

Verification of the signature involves computing $e(P, Q)$ for some points
$P, Q$. By the bilinearity of the pairing, if we have guessed the point $Q$
wrongly, we can obtain the value of $e(P,-Q) = e(P,Q)^{-1}$ by performing
an inversion rather than recompute another pairing.

Later we will see we can do better than this, and in effect check both
cases at once.

Note instead of discarding the $y$-coordinate, we can
replace it with a single bit signifying which solution of $y$ to take.

\section {Approaches to Finding Curves}

A randomly-chosen elliptic curve will have a large embedding
degree, which bestows it with resistance to the aforementioned discrete log
attack, but also makes it useless for pairing-based cryptography.
We must take one of two approaches to find curves suitable for pairings:

\begin{enumerate}
\item
Supersingular curves are guaranteed to have a small embedding degree,
and are easy to construct. They have been completely classified. Operations
on some of them can be highly optimized.
\item
By carefully tailoring the complex multiplication method of constructing
elliptic curves, we can produce curves of a certain embedding degree. 
\end{enumerate}

We note that supersingular hyperelliptic curves have also been considered,
and it is thought that using the hyperelliptic equivalent of the
complex multiplication method may lead to more curves with low
embedding degrees, though further research in this area is required.

Below we survey some types of curves used in practice for pairing-based
cryptography. To make them easier to refer to, we label them with letters.
The labelling is arbitrary, though we try to arrange them in the order
they appeared in cryptography publications.

\section {Supersingular Curves}

There are six families of supersingular curves, with embedding degree at
most six \cite{mov}. Let $q=p^m$. Let $t$ denote the trace of Frobenius.
Then the six classes can be described as follows.

\begin{enumerate}
\item
$k = 2$: $ t = 0$ and $E(\mathbb{F}_q) \cong \mathbb{Z}_{q+1}$.
\item
$k = 2$: $ t = 0$ and $E(\mathbb{F}_q) \cong \mathbb{Z}_{(q+1)/2} \oplus \mathbb{Z}_2$ and $q = 3 \pmod{4}$.
\item
$k = 3$: $ t^2 = q$ and $m$ is even.
\item
$k = 4$: $ t^2 = 2q$ and $p = 2$ and $m$ is odd.
\item
$k = 6$: $ t^2 = 3q$ and $p = 3$ and $m$ is odd.
\item
$k = 1$: $ t^2 = 4q$ and $m$ is even.
\end{enumerate}

Some curves in the first two classes are easy to describe, and are
extremely useful, as it is easy to find curves containing a
subgroup of any desired order.

The $k=4$ case requires $p=2$.
Many specialized optimizations exist for
operations in characteristic two fields,
but unfortunately at the same time specialized discrete logarithm attacks
exist \cite{coppersmith}, and we must use bigger fields to compensate for
this. 

TODO:how bad is it?

The $k=6$ case requires $p=3$. Again, we may apply
a host of specialized optimizations, but we must also be wary of
low-characteristic discrete logarithm algorithms.

\section {Type A Curves}

Let $q$ be a prime satisfying $q = 3 \pmod{4}$.
Let $E$ be the curve $y^2 = x^3 + x$. It can be shown \cite{silverman86}
that $E(\mathbb{F}_q)$ is supersingular and $k = 2$.

We have $\#E(\mathbb{F}_q) = q+1$, and $\#E(\mathbb{F}_{q^2}) = (q+1)^2$,
thus for any $P\in E(\mathbb{F}_q)$
the embedding degree of $G = \langle P \rangle$ is 2.

Note $-1$ is a quadratic nonresidue in $\mathbb{F}_q$ since $q = 3\pmod{4}$.
Consider the map given by
\[ \Psi(x, y) = (-x, i y) \]

Then $\Psi$ maps points of $E(\mathbb{F}_q)$ to points of
$E(\mathbb{F}_{q^2}) \setminus E(\mathbb{F}_q)$. Thus if $f$ denotes the
Tate or Weil pairing, then defining $e:G \times G \rightarrow \mathbb{F}_{q^2}$
by

\[ e(P,Q) = f(P, \Psi(Q)) \]

gives a bilinear nondegenerate map.

Setup for this type of pairing for a cryptosystem can be done as follows.

\begin{enumerate}
\item
An order $r$ is chosen, large enough to avoid generic discrete logarithm
attacks. Other properties may be desired. For some cryptosystems $r$ is
an RSA modulus. As we will see, careful choices of $r$ will speed up Miller's
algorithm substantially.
\item
Recall we require finite field discrete logarithm attacks on $\mathbb{F}_{q^2}$
to be impractical. Thus a random multiple of four $h$ is generated,
where $h$ is large enough to guarantee $(hr)^2$ is big enough to resist
finite field attacks. For example, if $r$ is 160 bits long, and we want
$q^2$ about 1024 bits long, then $h$ must be about 352 bits long.
\item
Next it is checked that $q = h r - 1$ is prime.
If not, we go back to the previous step.
\item
For some cryptosystems problems may arise if $r \nmid h$, but this occurs
with negligible probability for realistic parameters. Nonetheless, when toy
examples are constructed, this may need to be checked.
\end{enumerate}

If $h$ is constrained to be a multiple of $3$ as well, then cube roots are
extremely easy to compute in $\mathbb{F}_{q}$:
for all $x \in \mathbb{F}_q$ we see $x^{-(q-2)/3}$ is the cube root of $x$
(cube roots are necessarily unique since each element is a cube).
This may be desirable in some situations, and hardly affects the setup
algorithm.

\section {Type B Curves}

Let $q$ be a prime satisfying $q = 3 \pmod{4}$.
Let $E$ be the curve $y^2 = x^3 + 1$. Then
$E(\mathbb{F}_q)$ is supersingular and $k = 2$.

Again we have $\#E(\mathbb{F}_q) = q+1$ and $\#E(\mathbb{F}_{q^2}) = (q+1)^2$,
thus for any $P\in E(\mathbb{F}_q)$
the embedding degree of $G = \langle P \rangle$ is 2.

Consider the map given by
\[ \Psi(x, y) = (\zeta x, y) \]
where $\zeta$ is a primitive cube root of unity.

Then $\Psi$ maps points of $E(\mathbb{F}_q)$ to points of
$E(\mathbb{F}_{q^2}) \setminus E(\mathbb{F}_q)$. Thus if $f$ denotes the
Tate or Weil pairing, then defining $e:G \times G \rightarrow \mathbb{F}_{q^2}$
by

\[ e(P,Q) = f(P, \Psi(Q)) \]

gives a bilinear nondegenerate map.

As mentioned before, cube roots are easy to find since $q = 2 \pmod{3}$.
For this particular curve, it means we may easily generate points from
a $y$-coordinate: given any $y$, simply take $x = (y^2-1)^{-(q-2)/3}$.
This simplifies routines such as random point generation. Additionally,
points can be represented by their $y$-coordinate alone since there is
a unique $x$ for each value of $y$.

We may also ensure $-1$ is a quadratic nonresidue in $\Fq$ by
choosing $q = 3\pmod{4}$. This allows us to implement $\mathbb{F}_{q^2}$
as $\Fq[i]$, as above.

\section {Type C Curves}

Define the curves $E^{+} : y^2 = x^3 + 2 x + 1$ over $\F_{3^l}$
and $E^{-} : y^2 = x^3 + 2 x - 1$, also over $\F_{3^l}$.

Unlike all the other curves we consider,
we are working in a low characteristic field, and furthermore this
field does not have prime order.

It turns out
\[
\#E^+(\F_{3^l}) = \left \{ \begin{array}{rcl}
3^l + 1 + 3^{(l+1)/2} \mbox { when } l = \pm 1 \bmod 12 \\
3^l + 1 - 3^{(l+1)/2} \mbox { when } l = \pm 5 \bmod 12
\end{array} \right.
\]
and $\#E^-(\F_{3^l}) + \#E^+(\F_{3^l}) = 2(3^l + 1)$, that is
\[
\#E^-(\F_{3^l}) = \left \{ \begin{array}{rcl}
3^l + 1 - 3^{(l+1)/2} \mbox { when } l = \pm 1 \bmod 12 \\
3^l + 1 + 3^{(l+1)/2} \mbox { when } l = \pm 5 \bmod 12
\end{array} \right.
\]
and it is easily checked that both types of curve have embedding
degree 6 (that is, the order divide $3^{6l} -1$ in all cases).

For an $l = \pm 1, \pm 5$ and a
curve $E : y^2 = x^3 + 2 x \pm 1$ over $\mathbb{F}_{3^l}$
let $t$ be a root of $t^3 + 2t \pm 2 = 0$, and let $i$ be a
square root of $-1$. These both exist in $\mathbb{F}_{3^{6l}}$.
Define the map $\Psi$ by
\[ \Psi(x,y) = (-x + t, i y) \]
where $x, y$ are elements of $\F_{3^{6l}}$. Then $\Psi$ maps points
of $E(\F_{3^l})$ to points of $E(\F_{3^{6l}})$.

We arrive at a curve selection algorithm:

\begin{enumerate}
\item
Choose $l = \pm 1, \pm 5 \bmod 12$. To avoid Weil descent attacks,
$l$ should not have any small prime factors (say $3,5,7$).
\item
Compute $3^l + l \pm 3^{(l+1)/2}$ and check if either has a large prime
factor $r$.
\item
If so, set $E$ to the corresponding curve equation.
\end{enumerate}

It turns out there are not many suitable curves.

TODO: give table, mention coppersmith, cite weil descent paper

\section {Complex Multiplication}

Suppose we have integers $D, V, q, t$ satisfying the \emph{CM equation}
\[ D V^2 = 4 q - t^2 \]
such that $D = 0, 3 \bmod 4$ is positive and $q$ is prime.

View the Hilbert polynomial $H_D(x)$ as a polynomial in $\F_q[x]$,
and let $j \in \F_q$ be any root. For $j \ne 0, 1728$, set
$k = j / (1728 - j)$. Then the curve
\[ E: y^2 = x^3 + 3 k c^2 x + 2 k c^3 \]
has $j$-invariant $j$ for any nonzero $c \in \F_q$.

Set $c = 1$. Then $E$ has order $q + t + 1$ or $q - t + 1$.
Generate a random point $P$ of $E$ and check if
$(q-t+1)P = O$. If not, then the order must be $q + t + 1$,
and if we set $c$ to be some quadratic nonresidue in $\F_q$ the
curve will have order $q - t + 1$.

For $j = 0$ the curve has the form $y^2 = x^3 + k$ for some $k$.
For $j = 1728$ the curve has the form $y^2 = x^3 + k x$ for some $k$.
In these cases one can try different values of $k$ until a curve
with the correct order is found.

Thus given a solution the above equation, we may easily write down
a curve of order $q+t-1$ or $q-t-1$.

\section {Type D Curves}

Miyaji et al. describe a method for constructing ordinary elliptic
curves with embedding degree 3,4 or 6 \cite{mnt}. Moreover, they show
that in a sense, there are no other parametrizations that lead to curves
with these embedding degrees.

We first examine the embedding degree six case, which is the most useful.
The embedding degree three and four cases may also be useful in special
circumstances and we briefly describe them.

Consider the polynomials $q = x^2 + 1, t = \pm x + 1$.
It turns out that $q(x) + 1 - t(x) | q(x)^6 - 1$.

The CM equation becomes $D V^2 = 3 x^2 \pm 2 x + 3$. If we make
the substitution $U = 3x \pm 1$ then we have the generalized Pell equation
\[
U^2 - 3DV^2 = -8
\]

This suggests the following algorithm.

\begin{enumerate}
\item
Choose $D$ and solve the Diophantine equation
\[ D V^2 = 3 x^2 \pm 2 x + 3 \]
for $V, x$ by solving a Pell-type equation.
\item
Check that $q = x^2 + 1$ is prime
\item
Recall $t = \pm x + 1$.
Check that $q - t + 1 = q \mp x$ has a large prime factor $r$.
(Ideally it should be prime.)
\item
Check that $r$ does not divide $q^j = 1$ for any positive integer $j < k$.
(This can be omitted in practice as it is extremely unlikely this will
occur.)
\item
Use the CM method to construct a curve $E$ with order $q \mp x$.
\end{enumerate}

The resulting curve $E$ will have embedding degree 6.

We explain this method via the cyclotomic polynomials $\Phi_k(x)$.
Suppose $t = x + 1$.
Recall $\Phi_6(x) = x^2 - x + 1$, thus
$q = x + \Phi_6(x)$.
Then the order $n$ of the resulting
curve $E$ is $n = \Phi_6(x)$, thus
\[ q^6 - 1 = x^6 - 1 \bmod n \]

Recall $\Phi_6(x) | x^6 - 1$, hence $n$ must also divide $q^6 - 1$
giving an embedding degree of 6.

Now suppose $t = -x + 1$.
Recall $\Phi_3(x) = x^2 + x + 1$, thus
$q = -x + \Phi_3(x)$.
The order $n$ of $E$ is $n = \Phi_3(x)$, hence
\[ q^6 - 1 = (-x)^6 - 1 = x^6 - 1 \bmod n \]
Since $\Phi_3(x) | x^6 - 1$ the embedding degree of $E$ is 6.
Of course $\Phi_3(x) | x^3 - 1$ but $x^3 \ne (-x)^3$ in general
thus the embedding degree is not 3 (with high probability).

For an embedding degree of 4, we may also use cyclotomic polynomials.
Recall $\Phi_4(x) = x^2 + 1$. Then set $q = \pm x + \Phi_4(x)$
so that $t = \pm x + 1$. Since $n = \Phi_4(x)$ we have
\[ q^4 - x = (\pm x)^4 - 1 \bmod n \]
thus $n \mid q^4 - 1$.
(When the minus sign is chosen and $x$ replaced by $x+1$ 
we match the notation of Miyaji et al.\cite{mnt}.)

For $k=3$ Miyaji et al. give $q = 3 x^2 - 1$, $t = -1 \pm 3x$ where $x$ is
even.

\section {Type E Curves}

Curves of embedding degree 1 are easily constructed using the CM method.
Let $t = 2$, $D = 7$. Let $r$ be a positive integer
and suppose $q = 28 r^2 h^2 + 1$ is prime for some $h$.
Then the CM equation
\[
7 V^2 = 4 (28 (r h)^2 + 1) - 4
\]
is satisfied when we take $V = 4 r h$.

As $H_7(x) = x + 3375$, if we take $k = -3375 / (1728 - 3375)$ in $\F_q$
then the curve
\[
y^2 = x^3 + 3k x + 2k
\]
has order $q - 1$ or $q + 3$. Taking the twist if necessary,
find the curve with order $q-1$.

Note $r \mid q-1$ hence the Tate pairing on this curve can be computed
in $\F_q$. For most applications we choose $r$ to be prime but composite
orders can be chosen instead.

\section {Type F Curves}

By considering cyclotomic polynomials,
Barreto and Naehrig discovered a parametrization yielding embedding
degree 12 curves.

Let $q(x) = 36x^4 + 36x^3 + 24x^2 + 6x + 1$. Let $t(x) = 6x^2 + 1$.
Then if $D = 3$, the CM equation always has the solution
$V = 6x^2 + 4x + 1$. Furthermore,
it turns out $q(x) + 1 - t(x) | q(x)^{12} - 1$.
This suggests the following algorithm to generate curves:

\begin{enumerate}
\item
Pick an integer $x$ of a desired magnitude. It may be negative.
\item
Check if $q(x)$ is prime.
\item
Check if $n = q(x) - t(x) + 1$ has a large prime factor $r$.
(Ideally it should be prime.)
\item
Try different values of $k$ until a random point of
$y^2 = x^3 + k$ has order $n$
\end{enumerate}

Barreto and Naehrig recommend the last step be done as follows:
starting from $k = 1$, keep incrementing $k$ until $k+1$ is a quadratic
residue and $n(1,g) = O$, where $g$ is a square root of $k + 1$.

\section {Type G Curves}

The ideas of Miyaji et al. have been developed extensively\cite{alotofstuff}.
For example, methods exist to construct curves of any embedding degree
provided that one can tolerate a subgroup order $r$ that is
in general significantly smaller than the field order $q$.

Freeman unified many of the approaches in a single
framework\cite{freeman}, and also showed how to construct curves with
embedding degree 10, with $r$ and $q$ around the same size.
Freeman also notes that it seems unlikely that curves
with embedding degrees other than the ones described in this chapter
can be constructed using these techniques.

We summarize an algorithm due to Freeman and Scott:

Set $q(x) = 25x^4 + 25x^3 + 25x^2 + 10x + 3$ and
$t(x) = 10x^2 + 5x + 3$.

\begin{enumerate}
\item
Choose $D$ such that $D$ is squarefree and $D = 43, 67 \bmod 120$.
\item
Find solutions $(u,v)$ to the equation $u^2 - 15 Dv^2 = -20$.
\item
Check $u = \pm 5 \bmod 15$,
and set $x = (-5 \pm u)/15$.
\item
Check $q(x)$ is prime.
\item
Check $n = q(x) - t(x) + 1$ has a large prime factor $r$.
\item
Construct the corresponding curve $E$ using the CM method.
\end{enumerate}
%Multiply $u+v\sqrt{15 D}$ by a norm-one element of $\mathbb{Q}(\sqrt{15D})$
%and get next candidate

It can be shown that $n | q^{10} - 1$ thus the resulting curve has
embedding degree 10.

TODO: comparison table: symmetric? group sizes? embedding degree? ordinary?

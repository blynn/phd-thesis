\chapter{Summary of Contributions}

We briefly reiterate our main original contributions.

\section{Abstract definitions}

In Chapter 1 we gave abstract definitions of pairings that bridge the
gap between formal security proofs and bilinear maps used in practice.
We began with a symmetric definition, which was the first to appear in
the literature, and gradually extended it so that a greater variety of
pairings are available.

We presented examples of assumptions that pairing-based cryptosystems
rely on, and indicated how to modify them for different pairing definitions.

\section{The BLS Signature Scheme}

We exhibited a practical digital signature scheme with the shortest known
signature length at typical security levels.
This signature scheme, often referred to as the BLS signature scheme,
has many other desirable features~\cite{bls,bgls},
but they fall outside the scope of this text.
The BLS signature scheme is a pairing-based cryptosystem,
and many of these features rely heavily on
properties of bilinear maps. It is not known how to construct
signature schemes with similar advantages without using pairings.

\section{Constructing With Prescribed Embedding Degree}

We presented the
first published method for constructing cryptographically
useful pairings with any given embedding degree.
Inspired by the work of Miyaji et al.~\cite{mnt}, we use
cyclotomic polynomials to guarantee certain conditions are met,
yielding cryptographically-usedful
pairings with any given embedding degree~\cite{bals}.

\section{Optimizations}

In the last two chapters we showed how to
improve the running time of a pairing
by roughly a factor of four over a naive implementation.

Firstly, we proved that the classic definition of the
Tate pairing may be replaced by a simpler version that
can be evaluated in half the time~\cite{bakls}.

Secondly, using twist curves we can ignore denominators
in Miller's algorithm, doubling its speed~\cite{bals2}.
This technique is now so commonplace that it is
considered ``standard''~\cite{hsv}.

We also described methods for efficiently computing
the final powering, a costly operation that is required
to standardize coset representatives in the output group~\cite{bakls}.

Additionally, we gave more minor optimizations that together
substantially speed up a pairing computation, and pairing-based cryptosystems
in general~\cite{bakls, bals2}.

\chapter{Optimizing the Pairing}

Let $E$ be an elliptic curve over $\mathbb{F}_q$.
Let $P \in E(\mathbb{F}_q)$ be a point of prime order $r$.
Let $G = \langle P \rangle$, and let $k$ be the embedding degree of $G$.

\begin{theorem}
If $k > 1$ then
\[
e(P,Q) = f_P(Q)^{(q^k-1)/r}
\]
where $f_P$ is a function with divisor $r(P) - r(O)$.
\end{theorem}

\begin{proof}
Choose any point $R \in E(\mathbb{F}_q)$ that is not one of
$O, -P, Q, -Q, Q - P$,
and consider the function $f'_P$ that satisfies $(f'_P) = r(P+R) - r(R)$,
so that the Tate pairing can be computed by
\[
f'_P((Q)-(O))^{(q^k-1)/r}
\]
We have $f'(O) \in \mathbb{F}_q^*$ since it does not have a zero or pole
at $O$. Hence
$f'(O)^{(q^k-1)/r} = 1$ by Fermat's Little Theorem (we know $q-1$ must
divide $(q^k - 1)/r$ since $r\nmid q-1$) so the pairing can be computed using

\[ e(P,Q) = f'_P(Q)^{(q^k-1)/r} . \]

Let $g_R(X,Y)$ be the equation of a vertical line through $R$,
let $g_P(X,Y)$ be the equation of a vertical line through $P$,
and let $h(X,Y)$ be the equation of a line through $P+R$ and $-R$ (and hence
$-P$).
Thus we have

\[ (f'_P g_R^r g_P^r / h^r) = r (P) - r(O) = (f_P) \]

Each of $g_R(Q), g_P(Q), h(Q)$ lie in $\mathbb{F}_q^*$ since none
of the three lines go through $Q$ by choice of $R$.
Again by Fermat we have $f'_P(Q) = f_P(Q)$. (Actually, even if we
had chosen $g_R, g_P, h$ with coefficients in $\mathbb{F}_{q^k}$, our
argument still holds since each value is ultimately exponentiated by
$q^k-1$.)
Hence
\[e(P,Q) = f_P(Q)^{(q^k-1)/r} . \]
\end{proof}

\chapter {The Tate Pairing}

The Weil and Tate pairings \emph{torsion points}
as input, and in the case of the Weil pairing, both inputs are torsion points.
They can be defined using \emph{rational functions}.
They output an element of a finite field (that turns out to be a root of unity).

\section {Torsion Points}

Let $K$ be a finite field of characteristic $q$, so that
$K = \mathbb{F}_q^m$ for some natural number $m$.
Let $E$ be an elliptic curve defined
over $K$. Let $n=\#E(K)$
Suppose $P\in E(K)$ satisfies $r P = O$, so that $P$ has order $r$ or a
factor of $r$.
We call $P$ an \emph{$r$-torsion point}.
We denote the set of $r$-torsion points in $E(K)$ by
$E(K)[r]$.

It will be seen that for the curves we consider,
$n$ and $q$ are always coprime thus $r$ is also coprime to $q$
(since $r \mid n$).
The case when $r$ is not coprime
to $q$ can also be considered but leads to \emph{anomalous curves},
which are unsuitable for cryptography TODO: cite.

It can be proved~\cite{silverman} that for some integer $k \ge 1$,
$E(\Fqk)[r]$ contains exactly $r^2$ points and is isomorphic to
$\mathbb{Z}_r^+ \times \mathbb{Z}_r^+$,
and furthermore, for all $k' \ge k$ we have $E(\mathbb{F}_{q^{k'}} [r]
= E(\Fqk)[r]$.
Roughly speaking, there are no other $r$-torsion points beyond the
first $r^2$ points found, no matter how large the field extensions get.
(That is, if $\bar{K}$ is the algebraic closure of $K$ then $\#E(\bar{K})[r] = r^2$.)
We define $E[r] = E(\Fqk)[r]$, the set of
\emph{$r$-torsion points} of $E$.

It can be shown that $E[r]$ is the product of two cyclic groups of order $r$:

\[ E[r] \cong \mathbb{Z}_r \times \mathbb{Z}_r . \]

When $r=2$ this is easy to see: a point has order 2 if and only if it has
a zero $y$-coordinate. As $E$ is nonsingular
$x^3 + ax + b = 0$ has three distinct solutions, thus we can always find
some field $\Fqk$ where $E$ has
four points of order 2: the point $O$ and three points of the form
$(\alpha, 0)$ where $\alpha$ is a root of the cubic.

Moreover,
the line through any two of the finite points is simply the line $Y = 0$,
which certainly intersects $E$ at the other finite point.
The proof is much less trivial for general $r$.

The Weil pairing is a bilinear map
takes pairs of elements from $E[r]$ and outputs an $r$th root of unity
in $\Fqk$. The Tate pairing is similar but only the first
input is from $E(\Fq)[r]$.

\section {Rational Functions}

We study the behaviour of quotients of polynomials in two variables on
points on an elliptic curve.

We denote the ring of polynomials in two variables $X,Y$
with coefficients in $\Fqk$ by $\Fqk[X,Y]$. 

Let $E$ be an elliptic curve $Y^2 = X^3 + aX + b$ over $\Fqk$.
Write $E(X,Y)$ for the polynomial $Y^2 - X^3 - aX - b$.
It can be shown that a polynomial $f$ satisfies $f(X,Y) = 0$
whenever $E(X,Y) = 0$ if and only if $f(X,Y)$ is a multiple of $E(X,Y)$.

Thus define $\Fqk[E] = \Fqk[X,Y]_{E(X,Y)}$. In other words we look at all
possible polynomials and consider two of them to be
equivalent if they take the same values on all points of $E$.
Every element $f(X,Y)$ in $\Fqk[E]$ can be written in the form
$f_x(X) + Y f_y(X)$ for polynomials $f_x, f_y \in \Fqk[X]$ because
we may replace $Y^2$ by $X^3 + aX + b$.

Define $\Fqk(E)$ to be the field of fractions of $\Fqk[E]$.
Elements of $\Fqk(E)$ can be written in the form
$f(X,Y) / g(X,Y)$ where $f, g$ are polynomials in two variables $X, Y$
and $g$ is not a multiple of $E$.

\section {Curve Endomorphisms}

Given a point $P$ and integer $m$ consider the \emph{multiplication by $m$ map}
$[m]$ given by \[ P \mapsto m P . \]
If $P$ is viewed as a pair of variables $(X,Y)$, this map
can be written as $(f(X,Y), g(X,Y))$ for some rational functions $f,g$.
(Using the explicit formulas for point addition and doubling
given in the previous chapter, we can write down
explicit formulas for $f$ and $g$ for any $m$, but even for relatively
small $m$ the expressions are unwieldy.)

This is an example of a \emph{curve endomorphism}, because points of
$E(\Fqk)$ are mapped to points of $E(\Fqk)$. Another important example
is the $q$th power \emph{Frobenius map} $\Phi$,
given by $(X, Y) \mapsto (X^{q}, Y^{q})$. (In general one defines
$q^k$th power Frobenius maps but we only define the $q$th power map,
and if we need higher powers $k$ we shall write $\Phi^k$.)

\begin{theorem}
[Hasse]
\[ \Phi^k \circ \Phi^k - [t] \circ \Phi^k + [q^k] = [0] \]
where $t = q^k + 1 - \#E(\Fqk)$.
\end{theorem}

Since $E[r] \cong \mathbb{Z}_r \times \mathbb{Z}_r$ any curve endomorphism
that maps $E[r]$ to $E[r]$ (which is true for Frobenius and
multiplication-by-$m$ maps)
can be viewed as a $2\times 2$ matrix when restricted to $E[r]$.

In particular for a prime $r$ dividing $\#E(\Fq)$ the eigenvalues
of $\Phi$ are the solutions to $x^2 - t x + q \pmod{r}$.
The polynomial factorizes as $(x - q)(x - 1)$ thus any eigenvalues
must be $q$ or $1$. We can describe the corresponding eigenspaces.

Clearly $E(\Fq)[r]$ is a $1$-eigenspace, hence if $E[r] \subset E(\Fq)$
then $\Phi$ is simply the identity.

Otherwise consider the set $G = \{P \mid \tr P = O \}$
where the trace is defined by
\[ \tr P = P + \Phi(P) + ... + \Phi^{k-1}(P)  .\]

We know this set contains more than just the point at infinity
since for any $P \in E[r] \setminus E(\Fq)[r]$,
the point $P - \Phi(P)$ is nontrivial and has trace zero.
Straightforward computations show that $G$ is closed under the group laws,
and also show that $\Phi(G) = G$. Hence $G$ is an eigenspace.

As $G$ cannot be the $1$-eigenspace (which has already been accounted for;
$E(\Fq)[r]$ is the $1$-eigenspace), $G$ must be the $q$-eigenspace,
which implies $|G| = r$.

This result will be important later, so we will restate it:

\begin{theorem}
The set of points of trace zero
$G = \{P \mid \tr P = O \}$ is a cyclic group of order $r$,
and every $P \in G$ satisfies $\Phi(P) = q P$.
\end{theorem}

\section {Zeroes and Poles}

When we study a polynomial $f(X)$, the roots of $f(X)$ play an important
role. For example, if the roots are $\alpha_1,...,\alpha_n$ we may
instantly write down a polynomial with those roots, unique up to a constant:
$(X-\alpha_1)...(X-\alpha_n)$. Then when considering the quotient of
two polynomials $f(X)/g(X)$, we see the zeroes are the roots of $f$ and
the poles are the roots of $g$.

One way to view roots is to view them as places where the curve $Y = f(X)$
intersects with the curve $Y = 0$. We now generalize by replacing $Y=0$
with an elliptic curve.

When studying an element $f(X,Y) \in \Fqk[E]$, we take note of points of
intersection between $f(X,Y)$ and $E$.
We call these points of intersection \emph{zeroes} of $f$.

For example, consider the case $f(X,Y) = X-x$ for some $x\in\Fqk$.
Then $X-x$ intersects $E$ at one, two, or zero places depending
on whether $x^3 + a x + b$ is nonzero, and if so, whether it is a quadratic
residue or not.
We immediately see that zeroes in this case are harder to work with than
zeroes of a polynomial: given a polynomial $X - x$ we know it has
exactly one root $x$, while for elliptic curves, it may have 0,1, or 2
zeroes.

There are further complications. When $X - x$ intersects $E$ at one point,
it is analogous to a repeated root of a polynomial: it is in
fact a zero of multiplicity 2. The point at infinity also needs to be
considered. It turns out that, roughly speaking, any equation of a line
has a pole of mulitplicity 3 at $O$, but since $X - x$ passes through $O$,
$X-x$ has a pole of multiplicity 2 at $O$.

Recall an element of $\Fqk(E)$ can be written as $f(X,Y)/g(X,Y)$.
In this case the zeroes are the zeroes of $f$ and the poles of $g$,
and the poles are the poles of $f$ and the zeroes of $g$.

Note that knowing the zeroes and poles of the quotient of two
polynomial $f(X)/g(X)$ allows us to write down $f(X)/g(X)$ (up to multiplcation
by a constant factor). Suppose the zeroes and poles are $\alpha_1,...,\alpha_n$
with mulitplicites $a_1,...,a_n$ where $\alpha_i$ is a zero of multiplcity
$a_i$ if $a_i > 0$, and otherwise a pole of multiplicity $-a_i$. Then
we may write $f(X)/g(X)$ immediately by multiplying equations of
vertical lines correctly:

\[ f(X)/g(X) = (X-\alpha_i)^{a_i} \]

It turns out knowing the zeroes and poles of
a rational function on $E$ also allows us to instantly write its equation,
up to multiplication by a constant. We also build up the function by
multiplying together equations of lines, but not just using
vertical lines.

\section {Divisors}

Divisors are the standard way
to notate zeroes and poles and their multiplicities.
If we wish to record zeroes and poles $P_1,...,P_n$
of orders $a_1,...,a_n$ (where $P_i$ is a zero of order
$a_i$ if $a_i > 0$ and is a pole of order $-a_i$ otherwise), then
we may write:
\[ D = a_1\langle P_1\rangle + ... + a_n \langle P_n\rangle .\]

With this notation, if we add the divisors of two functions together,
then the resulting divisor reflects the zeroes and poles of the product
of the two functions.

For this chapter we break with tradition and use multiplicative
notation for divisors instead. In addition, we will omit an arbitrary
number of poles or zeroes at $O$. This is not problematic because we
will always be discussing principal divisors, thus the true multiplicity
of $O$ can be computed from the multiplicities at the finite points.

Let $D$ be the divisor with zeroes and poles $P_1,...,P_n$
of mulitplicities $a_1,...,a_n$.
We shall write $D$ as
\[ D = (P_1)^{a_1} ... (P_n)^{a_n} . \]
Furthermore if $D$ is a principal divisor, we will
omit as many $(O)$ terms as we like, as its real order
can be calculated simply by negating
of the sum of the orders of the other zeroes and poles.
(If we need to switch to conventional notation we will use angled brackets
rather than parentheses.)

On an elliptic curve, in our unorthodox notation:
\begin{description}
\item[Lines:]
a line $L$ through the points $P, Q$ has divisor
\[ (L) = (P)(Q)(-P-Q) . \]
\item[Tangents:]
when $Q = P$ we have a tangent line $T$ at $P$ with divisor
\[ (T) = (P)^2 (-2P) . \]
\item[Verticals:]
when $Q = -P$, we have a vertical line $V$ through the point $P$
with divisor
\[ (V) = (P)(-P) . \]
\end{description}

There are advantages to using this notation. Using additive group
notation for the elliptic curve and mutlplicative notation for the divisor
group emphasizes the difference between them and reduces the likelihood of
confusing the two. When rational functions are multiplied together,
their divisors can be multiplied together, which is arguably
more natural than adding them.
Leaving out as many $(O)$ terms as we choose allows us to focus less on
bookkeeping and more on the task at hand.
The correct multiplicity of $(O)$ can always be determined later.

We use
$(f)$ to denote the divisor of $f$, thus in our notation
for any rational functions $f, g$ we have $(f)(g) = (f g)$.

\section {The Weil Pairing}

Earlier papers on pairing-based cryptosystems advocated the Weil pairing.
We shall see that the related Tate pairing can be used instead.
Nonetheless we first describe the Weil pairing (as used in cryptography)
for historical reasons, and also because its simpler description
serves well as an introduction. The inputs to the Tate pairing are come
from different groups, and the definition involves quotient groups,
which can obscure the basic idea.

Let $E$ be an elliptic curve containing $n$ points over a field $\mathbb{F}_q$.
Let $G$ be a cyclic subgroup of $E(\mathbb{F}_q)$ of order $r$ with $r, q$
coprime. Let $k$ be the smallest positive integer such that $E(\Fqk)$
contains all of $E[r]$.

We define the Weil pairing
$f:E[r] \times E[r] \rightarrow \Fqk$ as follows.

For a pair of points $P, Q \in E[r]$,
let $f_P$ be a rational function with divisor $(f_P) = (P)^r$,
(In other words, we have $r$ zeroes at $P$ and $r$ poles at $O$.
In standard notation this is $r\langle P\rangle  - r\langle O\rangle$.)
and similarly let $f_Q$ be a rational function with divisor $(f_Q) = (Q)^r$.

Choose any $R, S \in E(\Fqk)$ such that $R \ne -P, O$ and $S \ne -Q, O$.
Define
\[ f(P,Q) = \frac{f_P(P+R)/f_P(R)}{f_Q(Q+S)/f_Q(S)} \]

Using Weil reciprocity it can be shown that this value is independent
of the choices for $R$ and $S$.

It is well-known that
\begin{enumerate}
\item
$f(a P, b Q) = f(P,Q)^{a b}$ for all $P, Q \in E[r]$ and all integers $a, b$.
\item
$f(P,P) = 1$ for all $P \in E[r]$.
\item
$f(P,Q) = 1$ for all $P \in E[r]$ if and only if $Q = O$.
\item
$f(P,Q) = 1$ for all $Q \in E[r]$ if and only if $P = O$.
\item
$f(P,Q) = f(Q,P)^{-1}$ for all $P,Q \in E[r]$.
\item
$f(\Phi(P),\Phi(Q)) = f(P,Q)^{q}$ for all $P,Q \in E[r]$.
where $\Phi$ denotes the Frobenius map.
\end{enumerate}

(The last property is usually stated more generally:
$f(\alpha(P),\alpha(Q)) = f(P,Q)^{\deg \alpha}$ for any nonzero
endomorphism $\alpha$.)

The second to last property, known as antisymmetry, is a property the
Tate pairing does not have. However it does not yet have
any use in cryptography.

Let $P$ be a generator of $G$. Suppose
we have a linear map
\[ \phi : E[r] \rightarrow E[r] \]
where $Q = \phi(P)$ linearly independent to $P$ and also generates
a group of order $r$.
Then defining $e:E[r]\times E[r]\rightarrow \Fqk$ by
\[ e(g,h) = f(g,\phi(h)) \]
for all $g, h \in G$
gives a symmetric bilinear nondegenerate map.

For efficiency, if possible we have $k > 1$, and choose
$G = E(\Fq)[r]$, namely the
group of points $P \in E(\Fq)$ satisfying $r P = O$.

\section {The Tate Pairing }

Let $E$ be an elliptic curve containing $n$ points over a field $\mathbb{F}_q$.
Let $G$ be a cyclic subgroup of $E(\mathbb{F}_q)$ of order $r$ with $r, q$
coprime. Let $k$ be the smallest positive integer such that $r \mid q^k - 1$.
For brevity write $K = \Fqk$.

The Tate (or Tate-Lichtenbaum) pairing
\[
e : E[r] \cap E(K) \times
E(K) / r E(K) \rightarrow
K^* / K^{*r}
\]
is defined as follows.

Let $f_P$ be a rational function with divisor $(f_P) = (P)^r$.
Choose an $R\in E(K)$ such that $R \ne P, P-Q, O, -Q$. The define
\[
f(P, Q) = f_P (Q + R) / f_P (R)
\]

It can be shown that the above value is independent of the choice of $R$,
and:
\begin{enumerate}
\item
$f(a P, b Q) = e(P, Q)^{a b}$ for all $P, Q, a, b$.
\item
$f(P,Q) = 1$ for all $P$ if and only if $Q = O$.
\item
$f(P,Q) = 1$ for all $Q$ if and only if $P = O$.
\item
$f(\Phi(P),\Phi(Q)) = f(P,Q)^{q}$ for all $P,Q \in E[r]$,
where $\Phi$ denotes the Frobenius map.
\end{enumerate}

The output of this pairing is some $x \in K^*$
that represents the coset $x K^{*r}$. To standardize the coset
representative, we exponentiate the output of the Tate pairing
by $(q^k - 1) / r$, which can take a substantial amount of time.

On the positive side, the second input to the Tate pairing is also a coset
representative. This means it can be any point of $E(K)$ and may
be of any order. (For example,
if the order of a point $Q$ is not a mulitple of $r$ then $Q$ represents
the coset $O + r E(K)$. Otherwise $Q$ represents some nonidentity
element.)

In contrast, when evaluating the Weil pairing we must ensure that the second
input $Q$ satisfies $r Q = O$.

\section {Miller's Algorithm}

We describe how to find a rational function $f_P$ with divisor
$(P)^r$ \cite{miller} where $P$ has order $r$.

For an integer $k$,
let $L_{kP}(X,Y)$ be an equation for a line through $P$ and $kP$,
and let
$V_{kP}(X,Y)$ be an equation for a vertical line through $kP$ (e.g.
$X - a$, where $kP = (a, b)$).
If there is no confusion we may drop the $P$ in $L_{kP}, V_{kP}$ and write
$L_k, V_k$ instead.

Notice we have
\[
\left (
\frac{L_k}{V_{k+1}}
\right )
= \frac{(P)(kP)(-(k+1)P)}{((k+1)P)(-(k+1)P)}
= \frac{(P)(kP)}{((k+1)P)}
\]
Define
\[ f_n = \prod_{k=1}^{n-1} \frac{L_k}{V_{k+1}} \]
Then
\[ (f_n) =
\frac{(P)(P)}{(2P)}
\cdot
\frac{(P)(2P)}{(3P)}
\cdots
\frac{(P)((n-1)P)}{(nP)}
\]
After cancellation we find $(f_n) = (P)^n / (nP)$. When $n = r$ this
is simply $(f_r) = (P)^r$, hence $f_P = f_r$.

We describe a more efficient method of computing $f_P$.
Observe $(f_k \cdot f_k) = {(P)^{2k}}/{(kP)^2}$
almost looks like $(f_{2k}) = {(P)^{2k}}/{(2kP)}$.
This suggests the following.

Let $T_{kP}(X,Y)$ (which we also write as $T_k$)
to be an equation for the tangent line at $kP$,
we have
\[
(f_k \cdot f_k \cdot T_k / V_{2k} ) = \frac{(P)^{2k}}{(kP)^2} \cdot
\frac{(kP)^2 (-2kP)}{(2kP)(-2kP)} = \frac{(P)^{2k}}{(2kP)} = (f_{2k}) .
\]
In short, we have
\[
f_{2k} = f_k^2 T_k / V_{2k}
\]
and from before
\[
f_{k+1} = f_k L_{k} / V_{k+1}
\]
thus to compute $f_r$ we can use an algorithm analogous to exponentiation via
repeated squaring.

For cryptographically useful curves the rational function $f_r$
is too long to compute and store. Instead, the function is only ever
evaluated when needed, and we evaluate as we perform the repeated
squaring.

The following algorithm, given points $P, Q$, computes
$f_r(Q)$.

\begin{enumerate}
\item
Set $x \leftarrow 1, Z \leftarrow P$.
Let the binary representation of $r$ be $b_t ... b_0$.
\item
For $i \leftarrow t-1, ..., 0$ do
    \begin{enumerate}
    \item
    Set $x \leftarrow x^2 \cdot T_Z(Q) / V_{2Z}(Q)$
    \item
    Set $Z \leftarrow 2Z$
    \item
    If $b_i = 1$ then
	\begin{enumerate}
	\item
	Set $x \leftarrow x \cdot L_{Z}(Q) / V_{Z+P}(Q)$
	\item
	Set $Z \leftarrow Z + P$
	\end{enumerate}
    \end{enumerate}
\end{enumerate}

After the algorithm terminates we have $x = f_r(Q)$ (and $Z = rP = O$).

There are complications however. The intermediate functions have zeroes
and poles that eventually cancel out. If we could derive algebraic expressions
for the rational function then we could manipulate them to remove these
zeroes and poles. But since we are evaluating the function as we go,
we cannot do this, and attempting to evaluate a function at a zero
or pole will cause problems.

Recall in our defintion of the Tate pairing we get to pick a point $R$.
Then we choose $R$ so that $Q+R$ and $R$ are never zeroes or poles
of the intermediate functions. One way to do this is to choose
$R$ at random (since there are only $O(t)$ bad choices for $R$).

Alternatively we may ensure the coordinates of $R$ do not lie in the
field that the functions are defined in, but rather in some field
extension, so that
$R$ cannot possibly be a zero or pole and neither can $Q + R$ (since the
zeroes and poles must be multiples of the point $P$).

Under some conditions there is a much simpler solution which we shall
discuss later.

Note also that care must be taken on the last iteration of the algorithm,
as there are degenerate cases. A vertical line at $O$ is simply a constant,
so does not needed to be computed. A line through $P$ and $-P$ is a vertical
line at $P$. If we ever need to compute a tangent at a point $P$ with a zero
$y$-coordinate, then in fact we are computing a vertical line at $P$.
The logic of the algorithm can be modified slightly to avoid checking
for these special cases.

TODO: example, say security will be detailed in next chapter

\chapter {Miller's Algorithm}

For this section we break with tradition and use multiplicative
notation for divisors. Let $D$ be some divisor
\[ D = a_0\langle P_0\rangle + ... + a_n \langle P_n\rangle . \]
We shall write $D$ as
\[ D = (P_0)^{a_0} ... (P_n)^{a_n} . \]
Furthermore if $D$ is a divisor of some rational function, we will
omit as many $(O)$ terms as we choose, as its true multiplicity
can be computed from the multiplicities of the other zeroes and poles.

On an elliptic curve, in our unconventional notation,
a line through the points $P, Q$ has divisor
\[ (P)(Q)(-P-Q) . \]
When $Q = P$ then this is a tangent line at $P$, and the divisor is
\[ (P)^2 (-2P) \]
and when $Q = -P$, we have a vertical line through the point $P$
giving the divisor
\[ (P)(-P) \]
(recall we can omit any $(O)$ terms).

There are advantages to using this notation. Using additive group
notation for the elliptic curve and mutlplicative notation for the divisor
group emphasizes the difference between them and reduces the likelihood of
confusing the two. When rational functions are multiplied together,
their divisors can be multiplied together, which is more natural than adding
them. Leaving out as many $(O)$ terms as we choose allows us to focus less on
bookkeeping and more on the task at hand.
The correct multiplicity of $(O)$ can always be computed later.

Given a rational function $f$,
$(f)$ denotes the divisor of $f$, thus in our notation
for any rational functions $f, g$ we have $(f)(g) = (f g)$.

We wish to find a rational function $f$ with divisor
$(P)^n$ (in standard notation this is $n\langle P\rangle  - n\langle O\rangle$).

For $k = 1,...,n$
let $g_k(X,Y)$ be an equation for a line through $P$ and $kP$,
and $h_k(X,Y)$ be an equation for a vertical line through $P$ (e.g.
$X - a$, where $P = (a, b)$). Notice we have
\[
\left (
\frac{g_k}{h_{k+1}}
\right )
= \frac{(P)(kP)(-(k+1)P)}{((k+1)P)(-(k+1)P)}
= \frac{(P)(kP)}{((k+1)P)}
\]
Define
\[ f = \prod_{k=1}^{n-1} \frac{g_k}{h_{k+1}} \]
Then
\[ (f) =
\frac{(P)(P)}{(2P)}
\cdot
\frac{(P)(2P)}{(3P)}
\cdots
\frac{(P)((n-1)P)}{(nP)}
\]
After cancellation we find $(f) = (P)^n / (nP)$. If $nP = O$ then this
is simply $(f) = (P)^n$.

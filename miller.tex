\chapter {The Tate Pairing}

\section {Rational Functions}

Consider the field $\Fqk(X,Y)_{E(X,Y)}$.

\section {Curve Endomorphisms}

Given a point $P$ and integer $m$ consider the \emph{multiplication by $m$ map}
$[m]$ given by \[ P \mapsto m P . \]
If $P$ is viewed as a pair of variables $(X,Y)$, this map
can be written as $(f(X,Y), g(X,Y))$ for some rational functions $f,g$.

This is an example of a \emph{curve endomorphism}, because points of
$E(\Fqk)$ are mapped to points of $E(\Fqk)$. Another important example
is the $q$th power \emph{Frobenius map} $\Phi$,
given by $(X, Y) \mapsto (X^{q}, Y^{q})$. (In general one defines
$q^k$th power Frobenius maps but we only define the $q$th power map,
and if we need higher powers $k$ we shall write $\Phi^k$.)

\begin{theorem}
[Hasse]
\[ \Phi^k \circ \Phi^k - [t] \circ \Phi^k + [q^k] = [0] \]
where $t = q^k + 1 - \#E(\Fqk)$.
\end{theorem}

Since $E[r] \cong \mathbb{Z}_r \times \mathbb{Z}_r$ any curve endomorphism
can be viewed as a $2\times 2$ matrix when restricted to $E[r]$.

In particular for a prime $r$ dividing $\#E(\Fq)$ the eigenvalues
of $\Phi$ are the solutions to $x^2 - t x + q \pmod{r}$.
The polynomial factorizes as $(x - q)(x - 1)$ thus any eigenvalues
must be $q$ or $1$. We can describe the corresponding eigenspaces.

Clearly $E(\Fq)[r]$ is a $1$-eigenspace, hence if $E[r] \subset E(\Fq)$
then $\Phi$ is simply the identity.

Otherwise consider the set $G = \{P \mid \tr P = O \}$
where the trace is defined by
\[ \tr P = P + \Phi(P) + ... + \Phi^{k-1}(P)  .\]

Then $G$ is group, and furthermore $\Phi(G) = G$ hence $G$ is an eigenspace.
As it cannot be a $1$-eigenspace it must be a $q$-eigenspace.

For this chapter we break with tradition and use multiplicative
notation for divisors. In addition, we will omit an arbitrary
number of poles or zeroes at $O$. This is not problematic because we
will always be discussing principal divisors, thus the true multiplicity
of $O$ can be computed from the multiplicities at the finite points.

\section {Divisors}

Let $D$ be divisor with zeroes and poles $P_1,...,P_n$ of mulitplicities
$a_1,...,a_n$. Rather than write
\[ D = a_1\langle P_1\rangle + ... + a_n \langle P_n\rangle \]
we shall write $D$ as
\[ D = (P_1)^{a_1} ... (P_n)^{a_n} . \]
Furthermore if $D$ is a principal divisor, we will
omit as many $(O)$ terms as we like, as its real order
can be calculated simply by negating
of the sum of the orders of the other zeroes and poles.
(If we need to switch to conventional notation we will use angled brackets
rather than parentheses.)

On an elliptic curve, in our unorthodox notation:
\begin{description}
\item[Lines:]
a line $L$ through the points $P, Q$ has divisor
\[ (L) = (P)(Q)(-P-Q) . \]
\item[Tangents:]
when $Q = P$ we have a tangent line $T$ at $P$ with divisor
\[ (T) = (P)^2 (-2P) . \]
\item[Verticals:]
when $Q = -P$, we have a vertical line $V$ through the point $P$
with divisor
\[ (V) = (P)(-P) . \]
\end{description}

There are advantages to using this notation. Using additive group
notation for the elliptic curve and mutlplicative notation for the divisor
group emphasizes the difference between them and reduces the likelihood of
confusing the two. When rational functions are multiplied together,
their divisors can be multiplied together, which is arguably
more natural than adding them.
Leaving out as many $(O)$ terms as we choose allows us to focus less on
bookkeeping and more on the task at hand.
The correct multiplicity of $(O)$ can always be determined later.

Given a rational function $f$,
$(f)$ denotes the divisor of $f$, thus in our notation
for any rational functions $f, g$ we have $(f)(g) = (f g)$.

\section {The Weil Pairing}

Earlier papers on pairing-based cryptosystems advocated the Weil pairing.
We shall see that the related Tate pairing has more advantages.
Nonetheless we first describe the Weil pairing (as used in cryptography)
for historical reasons, and also because it is simpler thus
serves well as an introduction. The inputs to the Tate pairing are come
from different groups, and the definition involves quotient groups.

Let $E$ be an elliptic curve containing $n$ points over a field $\mathbb{F}_q$.
Let $G$ be a cyclic subgroup of $E(\mathbb{F}_q)$ of order $r$ with $r, q$
coprime. Let $k$ be the smallest positive integer such that $E(\Fqk)$
contains all of $E[r]$.

We define the Weil pairing
$f:E[r] \times E[r] \rightarrow \Fqk$ as follows.

For a pair of points $P, Q \in E[r]$,
let $f_P$ be a rational function with divisor $(f_P) = (P)^r$,
(In standard notation this is $r\langle P\rangle  - r\langle O\rangle$.)
and similarly let $f_Q$ be a rational function with divisor $(f_Q) = (Q)^r$.

Choose any $R, S \in E(\Fqk)$ such that $R \ne -P, O$ and $S \ne -Q, O$.
Define
\[ f(P,Q) = \frac{f_P(P+R)/f_P(R)}{f_Q(Q+S)/f_Q(S)} \]

Using Weil reciprocity it can be shown that this value is independent
of the choices for $R$ and $S$.

It is well-known that
\begin{enumerate}
\item
$f(a P, b Q) = f(P,Q)^{a b}$ for all $P, Q \in E[r]$ and all integers $a, b$.
\item
$f(P,P) = 1$ for all $P \in E[r]$.
\item
$f(P,Q) = 1$ for all $P \in E[r]$ if and only if $Q = O$.
\item
$f(P,Q) = 1$ for all $Q \in E[r]$ if and only if $P = O$.
\item
$f(P,Q) = f(Q,P)^{-1}$ for all $P,Q \in E[r]$.
\item
$f(\Phi(P),\Phi(Q)) = f(P,Q)^{q}$ for all $P,Q \in E[r]$.
where $\Phi$ denotes the Frobenius map.
\end{enumerate}

(The last property is usually stated more generally:
$f(\alpha(P),\alpha(Q)) = f(P,Q)^{\deg \alpha}$ for any nonzero
endomorphism $\alpha$.)

The second to last property, known as antisymmetry, is a property the
Tate pairing does not have. However it does not yet have
any use in cryptography.

Let $P$ be a generator of $G$. Suppose
we have a linear map
\[ \phi : E[r] \rightarrow E[r] \]
where $Q = \phi(P)$ linearly independent to $P$ and also generates
a group of order $r$.
Then defining $e:E[r]\times E[r]\rightarrow \Fqk$ by
\[ e(g,h) = f(g,\phi(h)) \]
for all $g, h \in G$
gives a symmetric bilinear nondegenerate map.

For efficiency, if possible we have $k > 1$, and choose
$G = E(\Fq)[r]$, namely the
group of points $P \in E(\Fq)$ satisfying $r P = O$.

\section {The Tate Pairing }

Let $E$ be an elliptic curve containing $n$ points over a field $\mathbb{F}_q$.
Let $G$ be a cyclic subgroup of $E(\mathbb{F}_q)$ of order $r$ with $r, q$
coprime. Let $k$ be the smallest positive integer such that $r \mid q^k - 1$.
For brevity write $K = \Fqk$.

The Tate (or Tate-Lichtenbaum) pairing
\[
e : E[r] \cap E(K) \times
E(K) / r E(K) \rightarrow
K^* / K^{*r}
\]
is defined as follows.

Let $f_P$ be a rational function with divisor $(f_P) = (P)^r$.
Choose an $R\in E(K)$ such that $R \ne P, P-Q, O, -Q$. The define
\[
f(P, Q) = f_P (Q + R) / f_P (R)
\]

It can be shown that the above value is independent of the choice of $R$,
and:
\begin{enumerate}
\item
$f(a P, b Q) = e(P, Q)^{a b}$ for all $P, Q, a, b$.
\item
$f(P,Q) = 1$ for all $P$ if and only if $Q = O$.
\item
$f(P,Q) = 1$ for all $Q$ if and only if $P = O$.
\item
$f(\Phi(P),\Phi(Q)) = f(P,Q)^{q}$ for all $P,Q \in E[r]$,
where $\Phi$ denotes the Frobenius map.
\end{enumerate}

The output of this pairing is some $x \in K^*$
that represents the coset $x K^{*r}$. To standardize the coset
representative, we exponentiate the output of the Tate pairing
by $(q^k - 1) / r$.

On the positive side, the second input to the Tate pairing is also a coset
representative. This means it can be any point of $E(K)$ and may
be of any order. (For example,
if the order of a point $Q$ is not a mulitple of $r$ then $Q$ represents
the coset $O + r E(K)$. Otherwise $Q$ represents some nonidentity
element.)

In contrast, when evaluating the Weil pairing we must ensure that the second
input $Q$ satisfies $r Q = O$.

\section {Miller's Algorithm}

We describe how to find a rational function $f_P$ with divisor
$(P)^r$ \cite{miller} where $P$ has order $r$.

For an integer $k$,
let $L_{kP}(X,Y)$ be an equation for a line through $P$ and $kP$,
and let
$V_{kP}(X,Y)$ be an equation for a vertical line through $kP$ (e.g.
$X - a$, where $kP = (a, b)$).
If there is no confusion we may drop the $P$ in $L_{kP}, V_{kP}$ and write
$L_k, V_k$ instead.

Notice we have
\[
\left (
\frac{L_k}{V_{k+1}}
\right )
= \frac{(P)(kP)(-(k+1)P)}{((k+1)P)(-(k+1)P)}
= \frac{(P)(kP)}{((k+1)P)}
\]
Define
\[ f_n = \prod_{k=1}^{n-1} \frac{L_k}{V_{k+1}} \]
Then
\[ (f_n) =
\frac{(P)(P)}{(2P)}
\cdot
\frac{(P)(2P)}{(3P)}
\cdots
\frac{(P)((n-1)P)}{(nP)}
\]
After cancellation we find $(f_n) = (P)^n / (nP)$. When $n = r$ this
is simply $(f_r) = (P)^r$, hence $f_P = f_r$.

We describe a more efficient method of computing $f_P$.
Observe $(f_k \cdot f_k) = {(P)^{2k}}/{(kP)^2}$
almost looks like $(f_{2k}) = {(P)^{2k}}/{(2kP)}$.
This suggests the following.

Let $T_{kP}(X,Y)$ (which we also write as $T_k$)
to be an equation for the tangent line at $kP$,
we have
\[
(f_k \cdot f_k \cdot T_k / V_{2k} ) = \frac{(P)^{2k}}{(kP)^2} \cdot
\frac{(kP)^2 (-2kP)}{(2kP)(-2kP)} = \frac{(P)^{2k}}{(2kP)} = (f_{2k}) .
\]
In short, we have
\[
f_{2k} = f_k^2 T_k / V_{2k}
\]
and from before
\[
f_{k+1} = f_k L_{k} / V_{k+1}
\]
thus to compute $f_r$ we can use an algorithm analogous to exponentiation via
repeated squaring.

For cryptographically useful curves the rational function $f_r$
is too long to compute and store. Instead, the function is only ever
evaluated when needed, and we evaluate as we perform the repeated
squaring.

The following algorithm, given points $P, Q$, computes
$f_r(Q)$.

\begin{enumerate}
\item
Set $x \leftarrow 1, Z \leftarrow P$.
Let the binary representation of $r$ be $b_t ... b_0$.
\item
For $i \leftarrow t-1, ..., 0$ do
    \begin{enumerate}
    \item
    Set $x \leftarrow x^2 \cdot T_Z(Q) / V_{2Z}(Q)$
    \item
    Set $Z \leftarrow 2Z$
    \item
    If $b_i = 1$ then
	\begin{enumerate}
	\item
	Set $x \leftarrow x \cdot L_{Z}(Q) / V_{Z+P}(Q)$
	\item
	Set $Z \leftarrow Z + P$
	\end{enumerate}
    \end{enumerate}
\end{enumerate}

After the algorithm terminates we have $x = f_r(Q)$ (and $Z = rP = O$).

There are complications however. The intermediate functions have zeroes
and poles that eventually cancel out, but during the computation
attempting to evaluate a function at a zero or pole will cause problems.

Recall in our defintion of the Tate pairing we get to pick a point $R$.
Then we choose $R$ so that $Q+R$ and $R$ are never zeroes or poles
of the intermediate functions. One way to do this is to choose
$R$ at random (since there are only $O(t)$ bad choices for $R$).

Alternatively we may ensure the coordinates of $R$ do not lie in the
field that the functions are defined in, but rather in some field
extension, so that
$R$ cannot possibly be a zero or pole and neither can $Q + R$ (since the
zeroes and poles must be multiples of the point $P$).

Under some conditions there is a much simpler solution which we shall
discuss later.

Note also that care must be taken on the last iteration of the algorithm,
as there are degenerate cases. A vertical line at $O$ is simply a constant,
so does not needed to be computed. A line through $P$ and $-P$ is a vertical
line at $P$. If we ever need to compute a tangent at a point $P$ with a zero
$y$-coordinate, then in fact we are computing a vertical line at $P$.
The logic of the algorithm can be modified slightly to avoid checking
for these special cases.

TODO: example

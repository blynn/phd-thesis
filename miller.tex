\chapter {The Tate Pairing}

For this chapter we break with tradition and use multiplicative
notation for divisors. In addition, we will omit an arbitrary
number of poles or zeroes at $O$. This is not problematic because we
will always be discussing principal divisors, thus the true multiplicity
of $O$ can be computed from the multiplicities at the finite points.

\section { Divisors }

Let $D$ be divisor with zeroes and poles $P_1,...,P_n$ of mulitplicities
$a_1,...,a_n$. Rather than write
\[ D = a_1\langle P_1\rangle + ... + a_n \langle P_n\rangle \]
we shall write $D$ as
\[ D = (P_1)^{a_1} ... (P_n)^{a_n} . \]
Furthermore if $D$ is a principal divisor, we will
omit as many $(O)$ terms as we like, as its real order
can be calculated simply by negating
of the sum of the orders of the other zeroes and poles.
(If we need to switch to conventional notation we will use angled brackets
rather than parentheses.)

On an elliptic curve, in our unorthodox notation:
\begin{description}
\item[Lines:]
a line through the points $P, Q$ has divisor
\[ (P)(Q)(-P-Q) . \]
\item[Tangents:]
when $Q = P$ we have a tangent line at $P$ with divisor
\[ (P)^2 (-2P) . \]
\item[Verticals:]
when $Q = -P$, we have a vertical line through the point $P$
giving the divisor
\[ (P)(-P) . \]
\end{description}

There are advantages to using this notation. Using additive group
notation for the elliptic curve and mutlplicative notation for the divisor
group emphasizes the difference between them and reduces the likelihood of
confusing the two. When rational functions are multiplied together,
their divisors can be multiplied together, which is arguably
more natural than adding them.
Leaving out as many $(O)$ terms as we choose allows us to focus less on
bookkeeping and more on the task at hand.
The correct multiplicity of $(O)$ can always be determined later.

Given a rational function $f$,
$(f)$ denotes the divisor of $f$, thus in our notation
for any rational functions $f, g$ we have $(f)(g) = (f g)$.

\section {The Weil Pairing}

Earlier papers on pairing-based cryptosystems advocated the Weil pairing.
We shall see that the Tate pairing is similar but has more advantages.
Nonetheless we first describe the Weil pairing as used in cryptography
for historical reasons, and also because it is simpler thus
serves well as an introduction.

Let $E$ be an elliptic curve containing $n$ points over a field $\mathbb{F}_q$.
Let $G$ be a cyclic subgroup of $E(\mathbb{F}_q)$ of order $r$ with $r, q$
coprime. Let $k$ be the smallest positive integer such that $E(\Fqk)$
contains all of $E[r]$.

We define the Weil pairing
$f:E[r] \times E[r] \rightarrow \Fqk$ as follows.

For a pair of points $P, Q \in E[r]$,
let $f_P$ be a rational function with divisor $(f_P) = (P)^r$,
(In standard notation this is $r\langle P\rangle  - r\langle O\rangle$.)
and similarly let $f_Q$ be a rational function with divisor $(f_Q) = (Q)^r$.

Choose any $R, S \in E(\Fqk)$ such that $R \ne -P, O$ and $S \ne -Q, O$.
Define
\[ f(P,Q) = \frac{f_P(P+R)/f_P(R)}{f_Q(Q+S)/f_Q(S)} \]

Using Weil reciprocity it can be shown that this value is independent
of the choices for $R$ and $S$.

It is well-known that
\begin{enumerate}
\item
$f(a P, b Q) = f(P,Q)^{a b}$ for all $P, Q \in E[r]$ and all integers $a, b$.
\item
$f(P,P) = 1$ for all $P \in E[r]$.
\item
$f(P,Q) = 1$ for all $P \in E[r]$ if and only if $Q = O$.
\item
$f(P,Q) = 1$ for all $Q \in E[r]$ if and only if $P = O$.
\item
$f(P,Q) = f(Q,P)^{-1}$ for all $P,Q \in E[r]$.
\item
$f(\Phi(P),\Phi(Q)) = f(P,Q)^{q}$ for all $P,Q \in E[r]$.
where $\Phi$ denotes the Frobenius map.
\end{enumerate}

(The last property is usually stated more generally:
$f(\alpha(P),\alpha(Q)) = f(P,Q)^{\deg \alpha}$ for any nonzero
endomorphism $\alpha$.)

Let $P$ be a generator of $G$. Suppose
we have a linear map
\[ \phi : E[r] \rightarrow E[r] \]
where $Q = \phi(P)$ linearly independent to $P$ and also generates
a group of order $r$.
Then defining $e:E[r]\times E[r]\rightarrow \Fqk$ by
\[ e(g,h) = f(g,\phi(h)) \]
for all $g, h \in G$
gives a symmetric bilinear nondegenerate map.

TODO: example

\section {The Tate Pairing }

Let $E$ be an elliptic curve containing $n$ points over a field $\mathbb{F}_q$.
Let $G$ be a cyclic subgroup of $E(\mathbb{F}_q)$ of order $r$ with $r, q$
coprime. Let $k$ be the smallest positive integer such that $r \mid q^k - 1$.
For brevity write $K = \Fqk$.

The Tate (or Tate-Lichtenbaum) pairing
\[
e : E[r] \cap E(K) \times
E(K) / r E(K) \rightarrow
K^* / K^{*r}
\]
is defined as follows.

Let $f_P$ be a rational function with divisor $(f_P) = (P)^r$.
Choose an $R\in E(K)$ such that $R \ne P, P-Q, O, -Q$. The define
\[
f(P, Q) = f_P (Q + R) / f_P (R)
\]

It can be shown that the above value is independent of the choice of $R$,
and:
\begin{enumerate}
\item
$f(a P, b Q) = e(P, Q)^{a b}$ for all $P, Q, a, b$.
\item
$f(P,Q) = 1$ for all $P$ if and only if $Q = O$.
\item
$f(P,Q) = 1$ for all $Q$ if and only if $P = O$.
\item
$f(\Phi(P),\Phi(Q)) = f(P,Q)^{q}$ for all $P,Q \in E[r]$,
where $\Phi$ denotes the Frobenius map.
\end{enumerate}

TODO: example

\section {Miller's Algorithm}

We describe how to find a rational function $f$ with divisor
$(P)^r$ \cite{miller} where $P$ has order $r$.

For an integer $k$,
let $l_{kP}(X,Y)$ be an equation for a line through $P$ and $kP$,
and let
$v_{kP}(X,Y)$ be an equation for a vertical line through $kP$ (e.g.
$X - a$, where $kP = (a, b)$).
If there is no confusion we may drop the $P$ in $l_{kP}, v_{kP}$ and write
$l_k, v_k$ instead.

Notice we have
\[
\left (
\frac{l_k}{v_{k+1}}
\right )
= \frac{(P)(kP)(-(k+1)P)}{((k+1)P)(-(k+1)P)}
= \frac{(P)(kP)}{((k+1)P)}
\]
Define
\[ f_n = \prod_{k=1}^{n-1} \frac{l_k}{v_{k+1}} \]
Then
\[ (f_n) =
\frac{(P)(P)}{(2P)}
\cdot
\frac{(P)(2P)}{(3P)}
\cdots
\frac{(P)((n-1)P)}{(nP)}
\]
After cancellation we find $(f_n) = (P)^n / (nP)$. When $n = r$ this
is simply $(f_r) = (P)^r$.

We describe a more efficient method of computing $f_r$.
Observe $(f_k \cdot f_k) = {(P)^{2k}}/{(kP)^2}$
almost looks like $(f_{2k}) = {(P)^{2k}}/{(2kP)}$.
This suggests the following.

Let $t_{kP}(X,Y)$ (which we also write as $t_k$)
to be an equation for the tangent line at $kP$,
we have
\[
(f_k \cdot f_k \cdot t_k / v_{2k} ) = \frac{(P)^{2k}}{(kP)^2} \cdot
\frac{(kP)^2 (-2kP)}{(2kP)(-2kP)} = \frac{(P)^{2k}}{(2kP)} = (f_{2k}) .
\]
In short, we have
\[
f_{2k} = f_k^2 t_k / v_{2k}
\]
and from before
\[
f_{k+1} = f_k l_{k} / v_{k+1}
\]
thus to compute $f_r$ we can use an algorithm analogous to exponentiation via
repeated squaring.

For cryptographically useful curves the rational function $f_r$
is too long to compute and store. Instead, the function is only ever
evaluated when needed, and we evaluate as we perform the repeated
squaring.

The following algorithm, given points $P, Q$, computes
$f_r(Q)$.

\begin{enumerate}
\item
Set $x \leftarrow 1, R \leftarrow P$.
Let the binary representation of $r$ be $b_t ... b_0$.
\item
For $i \leftarrow t-1, ..., 0$ do
    \begin{enumerate}
    \item
    Set $x \leftarrow x^2 \cdot t_R(Q) / v_{2R}(Q)$
    \item
    Set $R \leftarrow 2R$
    \item
    If $b_i = 1$ then
	\begin{enumerate}
	\item
	Set $x \leftarrow x \cdot l_{R}(Q) / v_{R+P}(Q)$
	\item
	Set $R \leftarrow R + P$
	\end{enumerate}
    \end{enumerate}
\end{enumerate}

After the algorithm terminates we have $x = f_r(Q)$ (and $R = rP = O$).

\beforepreface
\prefacesection{Preface}
Pairing-based cryptography is a relatively young area of cryptography that
revolves around a particular function with interesting propreties.
It allows the construction of novel cryptosystems that are otherwise
difficult or impossible to assemble using standard primitives.

We first define a cryptographic pairing abstractly and show how it
can be used to build a signature scheme that is simple yet has many
desirable properties.

Next we examine how a pairing can be implemented in practice. The only known
mathematical setting where suitable pairings exist are groups on certain
families of curves. We focus on the simplest and most well-understood case:
elliptic curves.

We discuss methods for finding curves that yield cryptographic pairings,
and outline various optimizations that can be applied. We shall see that
pairing-based cryptosystems are quite practical and compare well against
traditional cryptosystems.

It is hoped that this work will be useful guide to implementing pairing-based
cryptography to a programmer who has experience with conventional
cryptosystems. While not comprehensive (notably absent are in-depth
discussion of the characteristic 3 case, pairings where
the subgroup size is necessarily significantly smaller than the field size,
and hyperelliptic curves),
the information contained herein should be enough to allow one to build
practical pairing-based applications from scratch.

Some background is required by the reader as
we do not review basic abstract algebra.
On the other hand, we introduce enough elliptic curve theory for
cryptographic purposes.

Most of the described algorithms and optimizations
appear in the PBC (Pairing-Based Cryptography) library, maintained by the
author, and available under the GNU Public License at 
http://crypto.stanford.edu/ibe/

\beforepreface
\prefacesection{Preface}
Pairing-based cryptography is a relatively young area of cryptography that
revolves around a particular function with interesting propreties.
It allows the construction of novel cryptosystems that are otherwise
difficult or impossible to assemble using standard primitives.

We shall first define a cryptographic pairing abstractly and show how it
can be used to build a signature scheme that is simple yet has many
desirable properties.

Next we examine how a pairing can be implemented in practice. The only known
mathematical setting where suitable pairings exist are groups on certain
families of curves. We focus on the simplest and most well-understood case:
elliptic curves.

We discuss methods for finding curves that yield cryptographic pairings,
and outline various optimizations that can be applied. We shall see that
pairing-based cryptosystems are quite practical and compare well against
traditional cryptosystems.

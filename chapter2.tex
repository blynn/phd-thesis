\chapter{Curve Selection}

Let $E$ be an elliptic curve defined over a finite field $K$.
Let $P \in E(K)$ be a point of prime order $r$. Let $G = \langle P \rangle$.

If $L$ is a field extension of $K$
that contains the $r$th roots of unity (so $r$ divides $|L| - 1$),
then the Tate pairing
on $G$ can be computed by performing operations in $L$.

On the other hand,
let $E[r]$ denote the group of points of order $r$.
The Weil pairing is defined on $E[r]$. For some field extension
$L'$ of $K$, $E(L')$ contains $E[r]$, and the Weil pairing can be
computed by performing operations in $L'$.

In order for the Tate pairing to be efficiently computable,
operations must be efficient in $L$.
Similarly, for the Weil pairing to be efficiently
computable, operations must be efficient in $L'$.

Thus we seek fields $L$ or $L'$ that are small enough so that field
operations are still fast.

It turns out that such a field $L'$ will always contain the $n$th
roots of unity, but the converse is not true: it is possible
for a field $L$ to contain the $r$th roots of unity but
at the same time part of $E[r]$ lies outside $E(L)$.

However, the converse is almost true:

\section {The Embedding Degree}

\begin{theorem} \cite{BK}
Let $E$ be an elliptic curve defined over $\mathbb{F}_q$.
Let $r$ be a prime dividing $N = \#(\mathbb{F}_q$) with
$r \nmid q-1$. Then $E(\mathbb{F}_{q^k})$ contains all $r^2$
points of order $r$ if and only if $r \mid q^k - 1$.
\end{theorem}

\begin{proof}
It is well-known that if $E(\mathbb{F}_{q^k})$ contains $E[r]$
then $r \mid q^k - 1$, even without assuming $r \mid N$ or
$r \nmid q-1$.

Conversely, suppose $k > 1$ and $r \mid q^k - 1$.
Let $\Phi$ denote the Frobenius map. Consider the subgroup $T$ of $E[r]$
consisting of all points of trace zero, that is
\[
T = \{ Q \in E[r] : \tr Q = Q + \Phi(Q) + ... + \Phi^{k-1}(Q) = O \}
\]
(The group $T$ may be explicitly constructed by using the map $P \mapsto
P - \Phi(P)$ on points of $E(\mathbb{F}_{q^k})$.)
Now we have $\Phi(T) = T$, and also $T$ is not contained
in $E(\mathbb{F}_q)$ by assumption.

Hence $T$ is an eigenspace of $\Phi$, but not the $1$-eigenspace. Since the
eigenvalues of $\Phi$ are $1$ and $q$, we see that $T$ must be the
$q$-eigenspace of $\Phi$ and hence
\[
\Phi^k(Q) = q^k Q = Q
\]
since $r | q^k - 1$. Thus $T$, like $E(\mathbb{F}_q)$ is fixed under
$\Phi^k$, and since these groups are linearly independent they generate all
of $E[r]$, implying that all of $E[r]$ is fixed under $\Phi^k$. Hence
$E[r] \subset E(\mathbb{F}_{q^k})$.
\end{proof}

The $k$ in the above theorem is referred to as the \emph{embedding degree}.

\begin{definition}
Let $E$ be an elliptic curve defined over $\mathbb{F}_q$.
Let $P \in E(K)$ be a point of order $r$.
Let $G = \langle P \rangle$.
The \emph{embedding degree} of $G$ is the smallest positive integer $k$
such that $r \mid q^k - 1$.
\end{definition}

Thus we have shown that if the embedding degree $k$ is greater than 1,
then both the Tate pairing and the Weil pairing may be computed by
performing field arithmetic in $\mathbb{F}_{q^k}$.

However, if the embedding degree $k$ is 1, the Tate pairing may
be computable in $\mathbb{F}_q$, but the Weil pairing may not be.
In practice this is not significant because we shall see we prefer
$k > 1$. However, the Tate pairing turns out to have other benefits.

For example, consider the curve $E: Y^2 = X^3 + X + 6$ which has
18 points over $\mathbb{F}_{19}$.
The point $R=(0,5)$ generates a cyclic group of order 18,
so the point $P = 6R = (12,13)$ generates a cyclic group of order 3.

We see $E[3]$ cannot be contained in $\langle R \rangle = E(\mathbb{F}_{19})$
since $E[3]$ is not cyclic, thus the Weil pairing cannot be computed.
(It turns out we must move to $\mathbb{F}_{19^3}$ to do so.)

In contrast, $3 \mid 19-1$ hence the Tate pairing can be
computed in $\mathbb{F}_{19}$. In fact, we find $e(P, P) = 7$, where
$e$ denotes the Tate pairing.

\section { Security Considerations }

Thus our goal is to find curves $E(\mathbb{F}_q)$ containing a subgroup
$G$ of prime order $r$ with embedding degree $k$ small enough so that
operations in $\mathbb{F}_{q^k}$ are efficiently computable.
All other things being equal, we prefer $q$ to be small as possible so
that operations in $G$ are fast as possible and can be represented by
as few bits as possible.

On such curves, the Tate pairing is always efficiently computable.
If $k > 1$ then the Weil pairing is also efficiently computable.

On the other hand, we must consider the lower bounds on $q$ and $q^k$.
Recall elliptic curves are attractive for cryptography
because unlike finite fields,
no specialized discrete logarithm algorithm is known for a general elliptic
curve.

Generally speaking, for a cryptographically useful elliptic curve
we need only ensure that $r$ is large enough to foil generic attacks
(e.g. Pollard rho, baby-step-giant-step).

However, a low embedding degree $k$ complicates matters. A bilinear map
sends points of an elliptic curve to elements of a finite field
$\mathbb{F}_{q^k}$. In particular, given a subgroup $G$ of $E(\mathbb{F}_q)$
of prime order $r$, by bilinearity and nondegeneracy we have
\[ \dlog (P, nP) = \dlog ( e(P,P), e(P, nP) )
= \dlog (e(P,P),e(P,P)^n) = n \]
for $n \in \{0,...,r-1\}$.
Thus if the discrete logarithm problem can be solved
in $\mathbb{F}_{q^k}$, then it can also be solved in $G$ \cite{mov, fr}.

Hence we must also ensure $q^k$ is large enough so that
finite field discrete logarithm algorithms such as index calculus are
infeasible in $\mathbb{F}_{q^k}$.

The curves used in pairing-based cryptography are special in other senses.
They are supersingular, or have complex multiplication. This possibly means
they are vulnerable, but no specific attacks for either case are known.

We have now described all necessary conditions on the sizes of
$r$ and $q^k$:

\begin{enumerate}
\item
$r$ must be a large enough prime so that generic discrete logarithm attacks
in a group of order $r$ are ineffective. Since $q \approx \#E(\mathbb{F}_q)$,
this places a similar lower bound on $q$.
\item
$q$ ought to be as small as possible, so that computations in $\mathbb{F}_q$
are as fast as possible.
\item
$q^k$ must be large enough so that finite field discrete logarithm attacks
in $\mathbb{F}_{q^k}$ are ineffective.
\item
$q^k$ must be small enough so that operations in $\mathbb{F}_{q^k}$
are efficient. $q^k$ should be small as possible so that operations
are as fast as possible.
\end{enumerate}

Of course, the first three statements are true for any cryptographically
useful elliptic curve, not just for pairing-based cryptography.

Currently it is considered acceptable to have $r$ about 160-bits.
As for $q^k$, 1024 bits is adequate for many applications, and calculations
in fields of this size can certainly be performed.
Ideally we have $r \approx \#E(\mathbb{F}_q) \approx q$
and hence $q$ is also about 160 bits. This gives an
embedding degree $k$ around $1024 / 160 = 6.4$.

We shall see how to construct curves of embedding degree 6 and 12.
For the near future, embedding degree 6 should be reasonable if used along
with an $r$ that is at lesat $170$ bits long,
and embedding degree 12 curves may be more desirable as time passes.

\section { Approaches to Finding Curves }

A randomly-chosen elliptic curve will have a large embedding
degree, giving resistance to the aforementioned discrete logarithm
attack, but also making it useless for pairing-based cryptography.
We must take one of two approaches to find curves suitable for pairings:

\begin{enumerate}
\item
Supersingular curves are guaranteed to have a small embedding degree,
and are easy to construct. They have been completely classified. Operations
on some of them can be highly optimized.
\item
By carefully tailoring the complex multiplication method of constructing
elliptic curves, we can produce curves of a certain embedding degree. 
\end{enumerate}

We note that supersingular hyperelliptic curves have also been considered,
and it is thought that using the hyperelliptic equivalent of the
complex multiplication method may lead to more curves with low
embedding degrees, though further research in this area is required.

\section { Supersingular Curves }

There are six families of supersingular curves, with embedding degree at
most six \cite{MOV}. Let $q=p^m$. Let $t$ denote the trace of Frobenius.
Then the six classes can be described as follows.

\begin{enumerate}
\item
$k = 2$: $ t = 0$ and $E(\mathbb{F}_q) \cong \mathbb{Z}_{q+1}$.
\item
$k = 2$: $ t = 0$ and $E(\mathbb{F}_q) \cong \mathbb{Z}_{(q+1)/2} \oplus \mathbb{Z}_2$ and $q = 3 \pmod{4}$.
\item
$k = 3$: $ t^2 = q$ and $m$ is even.
\item
$k = 4$: $ t^2 = 2q$ and $p = 2$ and $m$ is odd.
\item
$k = 6$: $ t^2 = 3q$ and $p = 3$ and $m$ is odd.
\item
$k = 1$: $ t^2 = 4q$ and $m$ is even.
\end{enumerate}

Some curves in the first two classes are easy to describe, and are
extremely useful, as it is easy to find curves containing a
subgroup of any desired order.

The $k=4$ case requires $p=2$.
Many specialized optimizations exist for
operations in characteristic two fields,
but unfortunately at the same time specialized discrete logarithm attacks
exist \cite{coppersmith}, and we must use bigger fields to compensate for
this. 

TODO:how bad is it?

The $k=6$ case requires $p=3$. Again, we may apply
a host of specialized optimizations, but we must also be wary of
low-characteristic discrete logarithm algorithms.

\section { Type A Curves }

Let $q$ be a prime satisfying $q = 3 \pmod{4}$.
Let $E$ be the curve $y^2 = x^3 + x$. It can be shown \cite{silverman86}
that $E(\mathbb{F}_q)$ is supersingular and $k = 2$.

We have $\#E(\mathbb{F}_q) = q+1$, and $\#E(\mathbb{F}_{q^2}) = (q+1)^2$,
thus for any $P\in E(\mathbb{F}_q)$
the embedding degree of $G = \langle P \rangle$ is 2.

Note $-1$ is a quadratic nonresidue in $\mathbb{F}_q$ since $q = 3\pmod{4}$.
Consider the map
$E(\mathbb{F}_q) \rightarrow E(\mathbb{F}_{q^2})$ given by
\[ \Psi(x, y) = (-x, i y) \]

Then $\Psi$ maps points of $E(\mathbb{F}_q)$ to points of
$E(\mathbb{F}_{q^2}) \setminus E(\mathbb{F}_q)$. Thus if $f$ denotes the
Tate or Weil pairing, then defining $e:G \times G \rightarrow \mathbb{F}_{q^2}$
by

\[ e(P,Q) = f(P, \Psi(Q)) \]

gives a bilinear nondegenerate map.

Setup for this type of pairing for a cryptosystem can be done as follows.

\begin{enumerate}
\item
An order $r$ is chosen, large enough to avoid generic discrete logarithm
attacks. Other properties may be desired.
For faster pairings, it is desirable to choose $r$ so that repeated squaring
algorithms can be optimized: e.g. choose $r$ to have a low Hamming weight,
or to be a Solinas number (i.e. $r$ has the form $2^a \pm 2^b \pm 1$).
For some cryptosystems $r$ is
an RSA modulus.
\item
Recall we require finite field discrete logarithm attacks on $\mathbb{F}_{q^2}$
to be impractical. Thus a random multiple of four $h$ is generated,
where $h$ is large enough to guarantee $(hr)^2$ is big enough to resist
finite field attacks. For example, if $r$ is 160 bits long, and we want
$q^2$ about 1024 bits long, then $h$ must be about 352 bits long.
\item
Next it is checked that $q = h r - 1$ is prime.
If not, we go back to the previous step.
\item
For some cryptosystems problems may arise if $r \nmid h$, but this occurs
with negligible probability for realistic parameters. Nonetheless, when toy
examples are constructed, this may need to be checked.
\end{enumerate}

If $h$ is constrained to be a multiple of $3$ as well, then cube roots are
extremely easy to compute in $\mathbb{F}_{q}$:
for all $x \in \mathbb{F}_q$ we see $x^{-(q-2)/3}$ is the cube root of $x$
(cube roots are necessarily unique since each element is a cube).
This may be desirable in some situations, and hardly affects the setup
algorithm.

\section { Type B Curves }

Let $q$ be a prime satisfying $q = 3 \pmod{4}$.
Let $E$ be the curve $y^2 = x^3 + 1$. Then
$E(\mathbb{F}_q)$ is supersingular and $k = 2$.

Again we have $\#E(\mathbb{F}_q) = q+1$ and $\#E(\mathbb{F}_{q^2}) = (q+1)^2$,
thus for any $P\in E(\mathbb{F}_q)$
the embedding degree of $G = \langle P \rangle$ is 2.

Consider the map
$E(\mathbb{F}_q) \rightarrow E(\mathbb{F}_{q^2})$ given by
\[ \Psi(x, y) = (\zeta x, y) \]
where $\zeta$ is a primitive cube root of unity.

Then $\Psi$ maps points of $E(\mathbb{F}_q)$ to points of
$E(\mathbb{F}_{q^2}) \setminus E(\mathbb{F}_q)$. Thus if $f$ denotes the
Tate or Weil pairing, then defining $e:G \times G \rightarrow \mathbb{F}_{q^2}$
by

\[ e(P,Q) = f(P, \Psi(Q)) \]

gives a bilinear nondegenerate map.

As mentioned before, cube roots are easy to find since $q = 2 \pmod{3}$.
For this particular curve, it means we may easily generate points from
a $y$-coordinate: given any $y$, simply take $x = (y^2-1)^{-(q-2)/3}$.
This simplifies routines such as random point generation.

We may also ensure $-1$ is a quadratic nonresidue in $\mathbb{F}_q$ by
choosing $q = 3\pmod{4}$. This allows us to implement $\mathbb{F}_q^2$
as $\mathbb{F}_q[i]$, as above.

\section { Complex Multiplication }

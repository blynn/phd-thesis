\chapter{Optimizing the Pairing}

\begin{theorem}
Let $E$ be an elliptic curve over $\mathbb{F}_q$.
Let $P \in E(\mathbb{F}_q)$ be a point of prime order $r$.
Let $G = \langle P \rangle$, and let $k$ be the embedding degree of $G$.

If $k > 1$ then
\[
e(P,Q) = f_P(Q)^{(q^k-1)/r}
\]
is a bilinear nondegenerate map,
where $f_P$ is a function with divisor $r\langle P\rangle - r\langle O\rangle$.
\end{theorem}

\begin{proof}
Choose any point $R \in E(\mathbb{F}_q)$ that is not one of
$O, -P, Q, -Q, Q - P$,
and consider the function $f'_P$ that satisfies $(f'_P) = r\langle P+R\rangle
- r\langle R \rangle$.
The Tate pairing can be computed by
\[
(f'_P(Q)/f'_P(O))^{(q^k-1)/r}
\]
We have $f'(O) \in \mathbb{F}_q^*$ since it does not have a zero or pole
at $O$. Hence
$f'(O)^{(q^k-1)/r} = 1$ by Fermat's Little Theorem (we know $q-1$ must
divide $(q^k - 1)/r$ since $r\nmid q-1$) thus
\[ e(P,Q) = f'_P(Q)^{(q^k-1)/r} . \]
Let $v_R(X,Y)$ be the equation of a vertical line through $R$,
let $v_P(X,Y)$ be the equation of a vertical line through $P$,
and let $l(X,Y)$ be the equation of a line through $P+R$ and $-R$ (and hence
$-P$).
Then we have
\[ (f'_P v_R^r v_P^r / l^r) = r \langle P \rangle - r \langle O \rangle
= (f_P) \]
None of $v_R, v_P, l$ have zeroes or poles at $Q$ by choice of $R$.
Sicne each of $v_R(Q), v_P(Q), l(Q)$ is ultimately exponentiated by $q^k-1$
we have $f'_P(Q) = f_P(Q)$.
(Alternatively we could appeal to Fermat's Little Theorem
again, since the lines can be chosen to have coefficients in $\F_q$.)
Hence
\[e(P,Q) = f_P(Q)^{(q^k-1)/r} . \]
\end{proof}

\section {Twist Curves}

Let $E : y^2 = x^3 + a x + b$ be an elliptic curve over $\mathbb{F}_q$,
and $P$ be a point of prime order $r$.
Suppose the embedding degree $k$ of $G = \langle P \rangle$ is even.

Let $d = k / 2$. Let $v$ be a quadratic nonresidue in $\mathbb{F}_{q^d}$,
so that $\mathbb{F}_{q^k} = \mathbb{F}_{q^d}[\sqrt{v}]$.
Define $E'$ to be the curve $y^2 = x^3 + v^2 a x + v^3 b$. $E'$ is a twist
of $E$.

Define the map $\Psi:E'\rightarrow E(\mathbb{F}_q^k)$ by
\[ \Psi(x,y) = (v^{-1}x, v^{-3/2}y) . \]

Let $f$ be the Tate pairing (the Weil pairing may be substituted).
Let $G' = E'(\mathbb{F}_{q^d})[r]$. Then we have a bilinear map
\[ e : G \times G' \rightarrow \mathbb{F}_{q^k} \]
given by $e(P, Q') = f(P, \Psi(Q'))$.

(Recall if $\#E(\mathbb{F}_{q^d} = q^d + 1 - c$ then
$\#E'(\mathbb{F}_{q^d}) = q^d + 1 + c$ and
$\#E(\mathbb{F}_{q^k}) = (q^{d} + 1 + c)(q^{d} + 1 - c)$.
$E(\mathbb{F}_{q^d})$ does not contain $E[r]$ otherwise the embedding
degree would be at most $d$, so $r | q^{d} + 1 + c$ and
there exist order $r$ points in $E'(\mathbb{F}_{q^d})$.)

In practical terms, this means most operations are performed in
$\mathbb{F}_{q^d}$  or $\mathbb{F}_q$. Computations are only performed in
$\mathbb{F}_{q^k}$ when a pairing is being evaluated.

\section {Denominator Elimination}
\section {The Tate Exponentiation}

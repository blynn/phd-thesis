\chapter{Optimizing the Pairing}

TODO: Miller:
For faster pairings, it is desirable to choose $r$ so that repeated squaring
algorithms can be optimized: e.g. choose $r$ to have a low Hamming weight,
or to be a Solinas number (i.e. $r$ has the form $2^a \pm 2^b \pm 1$).

TODO: My trick:
The curve $E$ should be transformed (using twist-like trick)
to the form $y^2 = x^3 - 3 a^2 x + b$ for some small $a$.
For some curves $a = 0$, and this speeds things up (proj double)
Otherwise ideally $a = 1$, diff. of two squares.
For $a$ equal to small const,
$aZ^2 =  Z^2 + ... + Z^2$ which is pretty cheap and
do $(X-aZ^2)(X+aZ^2)$.

\begin{theorem}
Let $E$ be an elliptic curve over $\mathbb{F}_q$.
Let $P \in E(\mathbb{F}_q)$ be a point of prime order $r$.
Let $G = \langle P \rangle$, and let $k$ be the embedding degree of $G$.

If $k > 1$ then
\[
e(P,Q) = f_P(Q)^{(q^k-1)/r}
\]
is a bilinear nondegenerate map,
where $f_P$ is a function with divisor $r\langle P\rangle - r\langle O\rangle$.
\end{theorem}

\begin{proof}
Choose any point $R \in E(\mathbb{F}_q)$ that is not one of
$O, -P, Q, -Q, Q - P$,
and consider the function $f'_P$ that satisfies $(f'_P) = r\langle P+R\rangle
- r\langle R \rangle$.
The Tate pairing can be computed by
\[
(f'_P(Q)/f'_P(O))^{(q^k-1)/r}
\]
We have $f'(O) \in \mathbb{F}_q^*$ since it does not have a zero or pole
at $O$. Hence
$f'(O)^{(q^k-1)/r} = 1$ by Fermat's Little Theorem (we know $q-1$ must
divide $(q^k - 1)/r$ since $r\nmid q-1$) thus
\[ e(P,Q) = f'_P(Q)^{(q^k-1)/r} . \]
Let $V_R(X,Y)$ be the equation of a vertical line through $R$,
let $V_P(X,Y)$ be the equation of a vertical line through $P$,
and let $L(X,Y)$ be the equation of a line through $P+R$ and $-R$ (and hence
$-P$).
Then we have
\[ (f'_P V_R^r V_P^r / L^r) = r \langle P \rangle - r \langle O \rangle
= (f_P) \]
None of $V_R, V_P, L$ have zeroes or poles at $Q$ by choice of $R$.
Sicne each of $V_R(Q), V_P(Q), L(Q)$ is ultimately exponentiated by $q^k-1$
we have $f'_P(Q) = f_P(Q)$.
(Alternatively we could appeal to Fermat's Little Theorem
again, since the lines can be chosen to have coefficients in $\F_q$.)
Hence
\[e(P,Q) = f_P(Q)^{(q^k-1)/r} . \]
\end{proof}

\section {Twist Curves}

Let $E : y^2 = x^3 + a x + b$ be an elliptic curve over $\mathbb{F}_q$,
and $P$ be a point of prime order $r$.
Suppose the embedding degree $k$ of $G = \langle P \rangle$ is even.

Let $d = k / 2$. Let $v$ be a quadratic nonresidue in $\mathbb{F}_{q^d}$,
so that $\mathbb{F}_{q^k} = \mathbb{F}_{q^d}[\sqrt{v}]$.
Define $E'$ to be the curve $y^2 = x^3 + v^2 a x + v^3 b$. $E'$ is a twist
of $E$.

Define the map $\Psi:E'\rightarrow E(\mathbb{F}_q^k)$ by
\[ \Psi(x,y) = (v^{-1}x, v^{-3/2}y) . \]

Let $P$ be any point of order $r$ in $E(\Fq)$ and
$Q'$ be any point whose order is a multiple of $r$ in $E'(\F_{q^d})$.
In practice a randomly chosen point of $E'(\F_{q^d})$ is likely to work.

Let $G = \langle P \rangle, H = \langle \Psi(Q)\rangle$. The
latter is a subgroup in $E'(\Fqk)$.

Let $e$ be the Tate pairing.
Then $e$ is a bilinear map.
We may subsitute the Weil pairing if additionally
$Q$ is a point of order $r$ not in $G$.

(Recall if $\#E(\mathbb{F}_{q^d} = q^d + 1 - c$ then
$\#E'(\mathbb{F}_{q^d}) = q^d + 1 + c$ and
$\#E(\mathbb{F}_{q^k}) = (q^{d} + 1 + c)(q^{d} + 1 - c)$.
$E(\mathbb{F}_{q^d})$ does not contain $E[r]$ otherwise the embedding
degree would be at most $d$, so $r \mid q^{d} + 1 + c$ and
there exist order $r$ points in $E'(\mathbb{F}_{q^d})$.)

In practical terms, this means most operations are performed in
$\mathbb{F}_{q^d}$  or $\mathbb{F}_q$. The $\Psi$ map and
other computations ain $\Fqk$ are only performed
when a pairing is being evaluated.

We may view this trick as a
generalization of the $\Psi$ map used in Type A curves.
(We have $k =2 , d = 1$.
The curve $E : y^2 = x^3 + x$ is equivalent to $E' : y^2 = x^3 + v^2 x$
because either $v$ or $-v$ is a quadratic residue.)

A dilemma arise with Type B curves however. The twist
of a curve $E : y^2 = x^3 + 1 $ is $E: y^2 = x^3 + v^3 $ for some
quadratic nonresidue $v$. We could work on these two curves but now
the pairing is no longer symmetric. On the other hand, if we insist on
a symmetric pairing, then although most computations can still be done
in $\Fq$, we cannot employ the optimization of the next section. For
many applications symmetry is not important, and efficiency is more important.

TODO: also don't have asymmetric pairing. need general bilinear map defn.
or do we?

\section {Denominator Elimination}

We use the same notation and assumptions as the previous section.
In other words, again let $E : y^2 = x^3 + a x + b$ be an elliptic curve
over $\mathbb{F}_q$,
and $P$ be a point of prime order $r$.
Suppose the embedding degree $k$ of $G = \langle P \rangle$ is even
and write $k = 2d$.

Suppose we want to find $f_P(Q)$.
In Miller's algorithm, to calculate the denominator of
$f_P(Q)$, we evaluate the equation of various vertical lines
at a point. In other words, we compute $x - a$ where $a$ is the
$x$-coordinate of the line and $x$ is the $x$-coordinate of $Q$.

In the computation of the Tate pairing, we also exponentiate by
$(q^k - 1)/r$ to standardize the coset representative. Observe
$q^d - 1$ divides $(q^k-1)/r$ because if $r$ divides $q^d - 1$ then
the embedding degree is at most $d$, not $k$.

Thus if both $x$ and $a$ lie in $\F_{q^d}$ then we have
$(x-a)^{q^d - 1} = 1$ so they can be omitted during Miller's algorithm.
This occurs when twist curves are used, hence we may simplify
the computation of $f_P(Q)$ as follows.

\begin{enumerate}
\item
Set $f \leftarrow 1$, and $Z \leftarrow P$.
Let the binary representation of $r$ be $b_t ... b_0$.
\item
For $i \leftarrow t-1$ to $0$
    \begin{enumerate}
    \item
    Set $f \leftarrow f^2 T_Z(Q)$ and $Z \leftarrow 2Z$
    \item
    If $b_i = 1$ then
	\begin{enumerate}
	\item
	Set $f \leftarrow f L_Z(Q)$ and $Z \leftarrow Z + P$.
	\end{enumerate}
    \end{enumerate}
\end{enumerate}

If $r$ is odd, the if condition is true during the last iteration and
the last multiplication is
\[ f \leftarrow f L_{(r-1)P}(Q) .\]
Since $(r-1)P = -P$, this is equivalent to $f \leftarrow f V_P(Q)$
and hence can be skipped since no vertical line computations are needed.
In this case we have $f_r = f_{r-1}$ and the logic can be simplified.

\section {The Tate Exponentiation}

The last step of a Tate pairing computation is
to exponentiate some quantity $a$ by
\[ \frac{q^k-1}{r} = r^{-1} \prod_{d\mid k} \Phi_d(q) \]
Since $k$ is the embedding degree, we have $r \mid \Phi_k(q)$ (and no
cyclotomic polynomial of smaller degree).

Then $a^{(q^k-1)/r}$ may be computed as follows:
\begin{enumerate}
\item
Compute $b = a^d $ where
\[ d = \prod_{d\mid k, d<k} \Phi_d(q) , \]
exploiting the identity $x^q = x$ for all $x \in \Fq$.
\item
Since
\[ c = \frac{\Phi_k(q)}{r} \]
is an integer, compute the output $b^c$
using a standard exponentiation algorithm.
\end{enumerate}

We describe the steps in detail
for $k = 2$. Suppose $\F_{q^2}$ has been implemented
as $\Fq[\alpha]$. Typically $\alpha = i = \sqrt{-1}$.
Then $q^2 - 1 = \Phi_2(q)(q-1)$ and
$r \mid \Phi_2(q) = q + 1$. Write $a = u + \alpha v$ where $u,v \in \Fq$.
We have
\[ b = a^{q-1} = (u + \alpha v)^q a^{-1} = \frac{u + \alpha^q v}{a} .\]
The constant $\alpha^q$ can be precomputed. Usually $\alpha$ is a square root
of some quadratic nonresidue in $\Fq$, so $\alpha^q = -\alpha$ and
this step is essentially a single division.

Then compute $b^{(q+1)/r}$ using a standard exponentiation algorithm
to obtain $a^{(q^2-1)/r}$. We have effectively halved the size of the exponent.

Let us also work through the $k = 6$ case. Suppose we have $\F_{q^6}$
implemented as $\Fq[\alpha]$. Then
\[ q^6 - 1 = \Phi_6(q) (q^4 + q^3 - q - 1)\]
(where $\Phi_6(q) = q^2 - q + 1$).
If $a = u_0 + u_1 \alpha + ... + u^5 \alpha^5$ we have
\[ b =
a^{q^4 + q^3 - q - 1}
= \frac{
(u_0 + u_1 \alpha^{q^4} + ... + u_5 \alpha^{5q^4})
(u_0 + u_1 \alpha^{q^3} + ... + u_5 \alpha^{5q^3})}
{
(u_0 + u_1 \alpha^q + ... + u_5 \alpha^{5q})a
}
\]
where each power of $\alpha^q$ can be precomputed. Then exponentiate $b$
by $(q^2 - q + 1)/r$ using a standard algorithm.
In this case we have shrunk the exponent to roughly one third its original
size.

TODO: collating divisons

TODO: char 3 optimizations

TODO: precomputation

TODO: pairing compression

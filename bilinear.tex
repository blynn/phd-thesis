\chapter{Bilinear Maps}

Asymmetric ciphers, or public-key cryptosystems, are perhaps
the most celebrated contribution of modern cryptography.
They certainly have had the most impact. It is hard to imagine
what the world would be like without their revolutionary
approach to key distribution.

All public-key cryptosystems in wide use today can trace their
roots to the Diffie-Hellman key exchange protocol \cite{diffie76new}
or the RSA cryptosystem \cite{RSA}.
The former depends on cyclic groups
with particular properties. (The latter, though
using similar arithmetic operations, relies on different principles.
For example, RSA uses groups that
are not cyclic and requires that the order of the group to be unknown
to the attacker.)

Roughly speaking, bilinear maps, or pairings,
give cyclic groups additional properties.
Initially, in the 1990s,
these additional properties were seen as detrimental
as they could be exploited to break cryptosystems \cite{mov, fr, fmr},
but it was later discovered
that they could also be exploited to build cryptosystems.
Rather than avoiding pairings, one can seek them out to construct new
schemes.

Boneh and Franklin's identity-based encryption scheme \cite{bf} is perhaps
the most famous early example of what could be achieved using
bilinear maps, though not the first \cite{sok, j}.
First described in 1984 \cite{shamir1984},
researchers were unable to build an identity-based encryption scheme by
conventional means for approximately
twenty years. 
Boneh and Franklin found an elegant solution using
bilinear maps~\cite{bf}. Extending the basic idea leads to
identity-based schemes with additional useful properties
(such as authenticated or hierarchical
identity-based encryption~\cite{l, hl}). More generally,
so many cryptographic applications of the pairing have been identified
this area of research is sometimes considered its own field:
pairing-based cryptography~\cite{thelounge}.

We note that an identity-based scheme based on quadratic residues
(and not bilinear maps)
has since been proposed \cite{cocks01}, albeit one that is signifcantly
less practical.
However, this is the exception rather than the rule. In general it is not
known how to find conventional
equivalents of a given pairing-based cryptosystem.

\section{Cyclic Groups}

Let $G = \langle g \rangle$ of prime order $r$.
Let $g$ be a generator of $G$
and let $x, y, z$ be integers in $[0, r-1]$.
Consider the following problems.

\begin{description}
\item [Discrete Log Problem.]
Given $g, g^x$, compute $x$.
\item [Computational Diffie-Hellman Problem.]
Given $g, g^x, g^y$, compute $g^{x y}$.
\item [Decisional Diffie-Hellman Problem.]
Given $g, g^x, g^y, g^z$, determine if $x y = z$.
\end{description}

A vast array of cryptosystems including the Diffie-Hellman key
exchange protocol can be built assuming one of these problems,
or one of many other related problems, is difficult to solve.

The discrete log problem is the most important, for if it could be
solved, all other related problems could also be solved.

\section{Formal Security Definitions}

For completeness we include formal definitions of difficult problems
commonly used by cryptographers, even though
we do not need them as we do not provide any security proofs. Our
focus is on the implementation of pairing-based cryptosystems, not
their correctness.

There are two flavours of definitions. One is asymptotic and the
other uses fixed security parameters.

We first describe the asymptotic definition. Roughly speaking, a problem
is hard if there is no efficient algorithm that can solve it.

We say a function $f:\mathbb{Z}_{\ge 0} \rightarrow \mathbb{R}_{\ge 0}$
is negligible if for all $c > 0$ there exists a $\lambda_0 > 0$ such that
for all $\lambda \ge \lambda_0$, $f(\lambda) < 1 / \lambda^c$.
Intuitively a function is negligible if it is smaller than any polynomial
function for sufficiently large inputs. Examples of negligible functions
are $2^{-\lambda}$ and $\lambda^{-\log\lambda}$, while $\lambda^{-10000000}$
is nonnegligible.

Let $\mathcal G = \{G_1, G_2, ... \}$ be a family of cyclic groups
where the order of $G_i$ is an $i$-bit number.
For an algorithm $A$
define the \emph{advantage} of $A$ to be
\[
Adv_A(t) = \Pr [ A(g, g^x) = x ]
\]
is negligible,
where $g$ is randomly chosen (uniformly) from $G_t$ and $x$ from
$[0, |G_t|-1]$.

Then we say that
the discrete log problem is hard in $\mathcal G$ if $Adv_A$ is negligible
for all probabilistic expected polynomial-time algorithms $A$.

Decisional variants of problems must be handled differently.
For example, we say the decisional Diffie-Hellman problem is hard if
for all probabilistic expected polynomial-time algorithms $A$
\[
Adv_A(t) = | \Pr [ A(g, g^x, g^y, g^z) = B(x, y, z) ] - 1/2 |
\]
is negligible,
where $g$ is randomly chosen from $G_t$, $x, y$ from $[0,r-1]$, and
$z$ from $[0,r-1]$ with probability 1/2 and set to $x y$ otherwise,
and where $B$ returns 1 if $x y = z$ and 0 otherwise.

The other kind of definition on the other hand is couched explicitly in
terms of running times and probabilities. It is useful when examining
groups of a particular size, and similarly in cases where there are no
families of infinite size (e.g. an asymptotic definition cannot be used
with the DES cipher since DES keys are always 56 bits long).

When using fixed security parameters, we say the discrete
log problem is $(t, \varepsilon)$-hard in a given
group $G$ if for all probabilistic $t$-time algorithms $A$ we have
\[ \Pr [ A(g, g^x) = x ] < \varepsilon \]
where $g$ is uniformly chosen from $G$ and $x$ from $[0, |G| -1]$.

The security definitions for other problems in either flavour
are similarly constructed and are omitted here.

Of course, there is always another requirement for cyclic groups
used in cryptography.
Efficient algorithms for multiplying group elements, inverting group elements,
and hashing to a group element must exist.
Exponentiating and choosing a random element should be efficient as well,
but this is implied by fast multiplication: exponentiation by
repeated squaring is efficient, and for random element generation
one can simply choose
a random integer and raise a generator by this power.

In practice, ``efficient''
means fast enough so that on the platform that the system is deployed
the user will not notice any slowdown. With pairing-based cryptography,
the bilinear map must also be efficient.

\section{Concrete Cyclic Groups}

There are two mathematical settings to choose from
when implementing cyclic groups: finite fields and elliptic curves.
We give a short summary here but will expound on elliptic curves later.

For finite fields, one picks a prime $n$ and uses
a subgroup $G$ of $\mathbb{Z}_n^*$
of prime order $r$, so the group operation is field multiplication.

An RSA cryptosystem arises when $n$ is instead chosen to be the product of two
large primes $P$, $Q$, in which case computations are still possible
in $\mathbb{Z}_n^*$ even if its order is unknown.

For elliptic curves, one takes an elliptic curve $E$ over some
finite field $K$ and takes some subgroup $G$ of the group of points
$E(K)$ with prime order $r$, so the group operation is point addition.

There is no ellitpic curve version of RSA. However, we shall encounter
a case which may potentially be confused with RSA: when a curve is
chosen such that its group of points has order $n = PQ$.

Finite fields were proposed first, and much has been published
on implementing finite fields and their potential weaknesses.
Of particular note is the discovery of subexponential discrete log
algorithms such as index calculus, resulting in recommendations that
field sizes be at least 1024 bits.

Elliptic curves were investigated later, and their use in cryptography
was at first viewed as experimental. However, they are now much less
controversial thanks to the efforts of many researchers. A myriad
of optimizations
and algorithms were discovered for elliptic curve cryptography,
and encouringly, a specialized elliptic curve discrete log algorithm
has yet to be found. In other words, the fastest known way to break discrete log
for a general elliptic curve is to use a algorithm for a generic
cyclic group (such as Pollard rho or lambda method) implying that a group
size of 160 bits is roughly equivalent to 1024-bit finite field security.

Thus currently elliptic curves present many advantages over finite fields.
Despite more complex algorithms for basic operations, they are much faster
and more compact due to a smaller group size.

We note that finite fields can be viewed as a special case of elliptic
curves due to a correspondence between certain singular curves and
multipicative groups of finite fields.

\section{Bilinear Maps}

We shall begin with the original and simplest definition of the pairing.

\section{\label{sec:symmetricpairing}The Symmetric Pairing}

Let $G, G_T$ be cyclic groups of prime order $r$.
Let $g$ be a generator of $G$.
A \emph{bilinear pairing} or \emph{bilinear map}
$e$ is an efficiently computable function
\[
e:G \times G \rightarrow G_T
\]
such that
\begin{enumerate}
\item
\emph{(Nondegeneracy)}
$e(g,g) \ne 1$
\item
\emph{(Bilinearity)}
$e(g^a, g^b) = e(g,g)^{a b}$ for all $a, b \in \mathbb{Z}$
\end{enumerate}

Note a symmetric bilinear map is completely defined by the value it
takes at $e(g,g)$. Essentially there is always exactly one bilinear map
for any given cyclic group: if
$e(g,g)=1$ then we have the degenerate case and the other $r-1$ maps are
equivalent up to a constant,
in the sense that if $e_1(g,g) = g_1$ and $e_2(g,g) = g_2$
then $e_2 = g_1^{-1} g_2 e_1$. The difficulty lies in finding such a map
that is efficiently computable.

We can immediately see the power of such a map: given
$g, g^x, g^y, g^z$, then by bilinearity and nondegeneracy,
$z = x y$ if and only if
$e(g, g^z) = e(g^x, g^y)$. In other words we can solve the Decisional
Diffie-Hellman problem.
Note some existing cryptosystems rely on the intractability of this
problem thus cannot be constructed on a cyclic group with
a computable pairing.

However, it is not known how to solve the Computational Diffie-Hellman
or discrete log problem using a pairing.

\section{The BLS Signature Scheme}

This abstract definition can easily be used to construct
a cryptosystem. We have just noted that if cyclic group with such a
bilinear map exists, then we have a group where the Computational
Diffie-Hellman
problem is thought to be hard, yet the Decisional variant is easy to solve
(sometimes called Gap Diffie-Hellman groups~\cite{op}) and
imply a signature scheme,
often referred to as the BLS Signature Scheme~\cite{bls}:

\begin{description}
\item[Setup.]
Choose a Gap Diffie-Hellman group $G$ of prime order $r$.
Publish a generator $g \in G$.

\item[Key Generation.]
Choose a random $x$ in $\{1,...,r-1\}$. Output
the public key $g^x$ and the private key $x$.

\item[Signing.]
Given a message $h \in G$, output $h^x$.

\item[Verify.]
Given a message-signature pair $h, \sigma$ and public key $g^x$,
check that $\langle{ g, h, g^x, \sigma}\rangle$ is a Diffie-Hellman tuple.
\end{description}

It can be shown that under the computational Diffie-Hellman assumption,
this signature scheme is secure against existential
forgery under a chosen-message attack in the random oracle model \cite{bls}.

Hence an abstract view of the pairing is all that is needed to build a
cryptosystem, and frequently this suffices. However,
we shall see that it can be useful to understand pairings in more detail.
For example, we need more facts to show the above signature scheme is
short, the original motivation for its proposal.

This signature scheme in fact has other useful properties, including
batched verification and allowing the
the simple construction of aggregate, ring and verifiably-encrypted
signatures\cite{bgls}.

\section{New Hardness Assumptions}

Classic problems have natural counterparts in pairing-based
cryptography. In the examples below, let $g$ be a generator
for a group $G$ of prime order $r$, and let $e$ be a bilinear map
on $G$.

\begin{description}
\item [Bilinear Diffie-Hellman Problem.]
Given $g,g^x,g^y,g^z$ compute $e(g,g)^{xyz}$
\item [Decisional Bilinear Diffie-Hellman Problem.]
Given $g,g^x,g^y,g^z,e(g,g)^w$, determine if $w = x y z$.
\item [$q$-Strong Diffie-Hellman Problem.]
Given $g,g^x,...,g^{(x^q)}$, compute $c, g^{1/(x_c)}$
for any $c\in\{1,...,r-1\}$.
\end{description}

Just as an algorithm that solves discrete log can be used to solve
the computational and decisional Diffie-Hellman problems, security reductions
exist between some of these new problems~\cite{jouxsurvey}.

The above signature scheme does not use any of these new assumptions,
instead using the pairing to solve the decisional Diffie-Hellman problem.
On the other hand some pairing-based cryptosystems not only require the
pairing, they also rely on one or more of the above problems to be hard.

Sometimes a conventional cryptosystem suggests
a pairing-based counterpart. For example,
the above pairing-based signature scheme and a ring signature scheme
due to Rivest et al. \cite{rst} were presented in adjacent sessions
at a conference. Someone taking the ideas from both talks would not have much
difficulty in synthesizing their results. Indeed,
a pairing-based ring signature scheme was presented about one
year later\cite{bgls}.

As before, both asymptotic and fixed-security-parameter formal security
definitions can be constructed for each of the above problems.

\section{\label{sec:asymmetricpairing}Loosening the Pairing Definition}

In practice, symmetric pairings can be instantiated by
using suitable supersingular elliptic curves.
However, in order to allow a wider range of curves to be used,
the definition must be modified. One early pairing-based cryptography
publication by Boneh, Lynn and Shacham
suggested the following~\cite{bls}:

Let $G_1, G_2, G_T$ be cyclic groups of prime order $r$.
Assume the Diffie-Hellman problem is hard in $G_1$.
Let $\phi : G_2 \rightarrow G_1$ be an efficiently computable group
isomorphism. Let $g_2$ be a generator of $G_2$.
Set $g_1 = \phi(g_2)$ (so $g_1$ generates $G_1$). 
A bilinear pairing $e$ is an efficiently computable function
\[
e:G_1 \times G_2 \rightarrow G_T
\]
such that $e(g_1,g_2) \ne 1$ and
$e(g_1^a, g_2^b) = e(g_1,g_2)^{a b}$ for all $a, b \in \mathbb{Z}$.

This version of the bilinear map is sometimes called
the \emph{asymmetric pairing}.

Note if a problem is hard $G_1$, it is also hard in $G_2$
due to existence of an efficiently computable
$\phi$. Thus some security proofs require only minimal changes. For example,
a scheme based on the CDH assumption in $G$ under
the symmetric pairing definition could now be based on the CDH assumption in
$G_1$.

This definition allows a greater variety of pairings to be used,
notably those constructed on ordinary curves.
However, it is advantageous to further loosen the
definition.

\subsection{A Problem With Hashing}

Suppose we have chosen an ordinary curve that does not yield a symmetric
pairing, and suppose that we wish to redesign
the cryptosystem to instead use an asymmetric pairing defined above.

It turns out that there is no
known method to hash to an element of $G_2$ (such that
its discrete log to some fixed base is unknown). We can hash to elements of
$G_1$ and $G_T$, and perform all other cryptographically useful
operations in $G_2$ such as picking a random element, multiplication
and inversion. We can even map an element of $G_2$ to $G_1$.
It is simply that we cannot hash a string
to an element of $G_2$.

For some cryptosystems, this issue can complicate their design,
and in extreme cases we may have to give up and stick to symmetric pairings
only.

However, with a more general pairing definition, we can use ordinary curves
and still be able to hash to $G_2$.

\section{\label{sec:generalpairing}The General Bilinear Pairing}

To enable certain optimizations (see Section~\ref{sec:denomelim}) and
to permit hashing to any element of any group including $G_2$,
we prefer the following definition.

Let $r$ be a prime.
Let $G_1, G_T$ be cyclic groups of order $r$.
Let $G_2$ be a group where each element has order dividing $r$.
In particular $G_2$ is not necessarily cyclic.
Again we use multiplicative group notation.
A bilinear pairing $e$ is an efficiently computable function
\[
e:G_1 \times G_2 \rightarrow G_T
\]
such that
\begin{enumerate}
\item
\emph{(Nondegeneracy)}
$e(g_1,g_2) = 1_{G_T}$ for all $g_2 \in G_2$ if and only if $g_1 = 1_{G_1}$,
and similarly 
$e(g_1,g_2) = 1_{G_T}$ for all $g_1 \in G_1$ if and only if $g_2 = 1_{G_2}$.
\item
\emph{(Bilinearity)}
for all $g_1 \in G_1$ and $g_2 \in G_2$:
$e(g_1^a, g_2^b) = e(g_1,g_2)^{a b}$ for all $a, b \in \mathbb{Z}$
\end{enumerate}

In this setting the hardness assumptions must be altered further.
Depending on the scheme, we may have to assume certain problems are hard in
both $G_1$ and $G_2$, or a combination of the two. For example, we may
need to assume that given $g_1, g_1^x \in G_1$ and $g_2 \in G_2$,
there is no efficient algorithm to compute $g_2^x$. (This particular
example is sometimes referred to as
the co-Computational Diffie-Hellman assumption~\cite{bls}.)

We have not required $r$ to be prime. A composite group
order is useful for some cryptosystems\cite{bgn},
but we must be aware that in this case even if $g_1$ and $g_2$ have order
$r$, $e(g_1, g_2)$ may not be a generator of $G_T$, but rather a generator
of some subgroup of $G_T$ whose order is a factor of $r$.

When $r$ is composite, it is incorrect to say all nondegenerate pairings are
equivalent up to a constant. Indeed, if $d > 1$ is a divisor of $r$
and $g_1, g_2$ generators of $G_1, G_2$ respectively, then a bilinear map
that takes $(g_1, g_2)$ to a $d$th root of unity is still nondegenerate.
Since there is more than one choice for $d$ for a composite $r$,
these maps cannot be equivalent up to a constant. They are however still
completely determined by $e(g_1, g_2)$ where $g_1, g_2$ are generators.
We present a concrete example later in Section~\ref{sec:compositeexample}.

This also means that $e(g_1, g_2) = 1$ does not imply at least
one of $g_1, g_2$ is the identity element, in contrast to the case when
$r$ is prime.

Thus some care is needed when dealing with composite $r$.
Facts about the pairing that are true for prime $r$ do not always carry over.
We shall encounter a concrete example of this phenomenon later.

Note when $r$ is chosen to be a product of two primes, the resulting scheme
is still substantially different to RSA as the groups are cyclic and of
known order.

\section{Exponentiation as a Bilinear Pairing}

Given some integer $r > 1,$
let us take $G_1 = G_T = \mathbb{Z}_r^*$ and $G_2 = \mathbb{Z}_{r-1}^+$.
Define $e:G_1\times G_2 \rightarrow G_T$ by $e(g, a)= g^a$.

Then it is easily seen that if we relax the condition that the groups
are cyclic, then exponentiation satisfies the definition of a bilinear pairing.
Thus we may view all discrete log and RSA cryptosystems
as pairing-based cryptosystems, though there are significant differences
in this setting. For example, the discrete log problem is easy
in $G_2$ and $G_1 = G_T$.

Classic problems can be restated in terms of
pairings. For example:

\begin{description}
\item[Discrete Log Problem.]
Given $g \in G_1, c \in G_T$, find $a \in G_2$ such that $e(g,a) = c$.
\item[RSA Problem.]
Given $a \in G_2, c \in G_T$, find $g \in G_1$ with
$e(g,a) = c$.
\item[Strong RSA Problem.]
Given $c \in G_T$, find $g \in G_1, a\in G_2$ with
$e(g,a) = c$
\end{description}

Thinking of conventional cryptosystems as pairing-based cryptosystems
can inspire new ideas.

\section{Choosing a Definition}

The symmetric pairing is often selected
as it allows simpler and briefer mathematical statements and definitions,
and the symmetry can be useful.
However, there are few families of groups that allow such a pairing.

In practice, one can usually trivially modify such
schemes and their corresponding
security proofs to use a more general pairing definition,
so that other groups can be used. For example, one may have to replace the
the Computation Diffie-Hellman assumption with
the co-Computation Diffie-Hellman problem. Or perhaps
a minor reworking allows the cryptosystem to function when $G_2$
is not cyclic.
Of course, the schemes that cannot be changed have a limited selection
of pairings.

We shall see there are other subtle caveats.
For example, if we use a symmetric pairing, then in typical use,
elements will be 512 bits long.
If we forgo symmetry and use two groups $G_1, G_2$,
then it turns out we may achieve a 160-bit $G_1$ and a 320-bit $G_2$,
though pairings will be slower.

\section{Concrete Bilinear Maps}

There is only one known setting for cryptographically useful bilinear maps,
namely elliptic curves with an efficiently computable
Tate (or similar) pairing.

Strictly speaking, one can use hyperelliptic
curves where much of theory can be generalized, but this is beyond
our scope. To the author's knowledge, there are no implementations
of pairing-based cryptosystems constructed using hyperelliptic curves.
Additionally, at present only \emph{supersingular} hyperelliptic
curves~\cite{galbraith,rubinsilverberg} can be used but there is no compelling
reason to do so.
Hyperelliptic equivalents of other elliptic curve families
used for pairing-based cryptography have yet to be discovered.

In practical terms, this means $G$, or $G_1$ and $G_2$ are always
groups of points on
an elliptic curve, and that $G_T$ is always a subgroup of a multiplicative group
of a finite field.

Mathematicians have traditionally employed additive notation for the
group of points on an elliptic curve (since it is
Abelian). As a result, publications in the area often use additive group
notation along with uppercase letters for elements for the input groups.

We have used multiplicative notation in this chapter to highlight the
Diffie-Hellman heritage of pairing-based cryptography. This notation also
suggests the right way to compare elliptic curve and finite field operations:
point multiplications should be compared with exponentiations, not
integer multiplications. Later we will switch to additive notation when
discussing the underlying mathematics.

When deploying a pairing-based cryptosystem one must select a type of
curve to use. The correct choice depends on several factors,
which we discuss in later chapters.

We conclude with a toy example of the BLS signature scheme.
Let $q = 427211363219$.
Consider the curve $E : y^2 = x^3 + x$ over
$\Fq$. Then $E(\Fq)$ contains a subgroup
$G$ of order $r = 524353$ (which is a prime factor of $427211363219 + 1$).
Take a random generator
\[ g = (359473638793, 293998693014) \]
for the system parameter, and a random secret key
\[ x = 66995 . \]
Then the corresponding public key is
\[ g^x = (166505572345, 357692656519) . \]
Suppose a message hashes to
\[ h = (370499138522, 78458612837) . \]
Then the corresponding signature is
\[ \sigma = h^x = (278647014375, 78919786700) . \]
To verify, we must check that $e(g, \sigma) = e(g^x , h)$
which turns out to be
\[ 85791756805 + 99975533880i \]
(an element of $K[i]$ that has order $r$).
The exact value of last result may vary
from implementation to implementation (the pairing is unique up to a
constant and different algorithms may result in different constants),
but both sides of the equation will still match.

The order $r$ in the above example is roughly 20 bits in length,
while the order $q$ of the field, and the actual space taken up by
a (compressed) elements of $G_1$ and $G_2$ is roughly 40 bits. In real life,
for the particular elliptic curve we chose,
one would require a 160-bit $r$ and 512-bit $q$
for secure signatures.

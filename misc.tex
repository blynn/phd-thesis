\chapter{Miscellaneous Optimizations}

Optimizing a pairing-based cryptosystem can be quite involved:

\begin{enumerate}
\item
Much effort must be devoted to speeding up
integer and finite field arithmetic operations, as they are present
in large amounts at the lowest level. For best results, platform-dependent
hand-coded assembly should be used.
Fortunately a lot of research has been conducted in this area,
and many implementations exist, some of them freely available.
\item
Elliptic curve operations should also be fast. Though less studied,
publications on this subject exist~\cite{bss}.
\item
Speeding up the pairing has only been a high priority relatively recently,
and is one of the main focuses of this work. We have devoted an entire
chapter to this topic.
\item
Sometimes modifications to a scheme will garner savings by
permitting larger amounts of precomputation, exploiting pairing
properties, moving exponentiations to cheaper groups, compressing points or
pairings, and so on.
\item
Needless to say, each of these classes of optimizations cannot be viewed
in isolation. An optimization that may not usually help could be beneficial
when the behaviour of the whole system is taken into account.
Similarly, one optimization could adversely affect another.

For one example, see the below
discussion on Montgomery representation.
\end{enumerate}

\section{Using Existing Libraries}

In all the author's pairing implementations, to avoid loss of portability
and assembly coding (as well as the need to learn platform-specific
low-level tricks), pre-existing multiprecision arithmetic libraries
were utilized.

Unfortunately, there does not seem to be a freely available library for
finite fields. Rather, libraries dealing with arbitrary-precision integers
are popular. Although it is easy to graft finite field routines on top
of any such library, there can be some losses.

For example, the GMP library has a highly optimized modular exponentiation
routine that uses Montgomery reduction (see below),
but expects the input not to be
in Montgomery form, and also outputs normal integers.

Hence if numbers are already internally stored in Montgomery form,
they must be converted to a normal integer first, and then converted
back to Montgomery form afterwards. Including the unwanted conversions in
GMP itself gives a total of four unnecessary conversions.

Another example is that an external library may be able to handle integers
of arbitrary length, but a particular deployment of a cryptosystem is
concerned only with integers in a certain range. Routines that know
in advance how many bytes their input integers take can be faster than
a more general routine.

\section{All or Nothing}

The nature of cryptography in a finite field requiring $n$ bytes per element
leads to the value of an integer data structure
often containing either zero, or an integer roughly $n$ bytes in length.
This is because many cryptographic operations can be viewed as a series of
arithmetic operations on $n$-byte numbers uniformly chosen at random.

One approach to exploit this is to attach a flag to every integer variable
that signifies when the variable contains the zero value. Then operations
such as addition and multiplication check this flag: if one of the input
variables is zero, the output is trivial to determine and no looping
is required, otherwise the computation should proceed on all $n$ bytes as
with high probability each of the $n$ bytes will be nonzero and thus
contribute to the result.

In some cases it may help to test for one as well.

\section{Montgomery Reduction}

For RSA or a discrete-log system in a finite field, it is recommended
only to use Montgomery reduction during an exponentiation
as the time lost from switching back and forth between representations
does not compensate for the savings gained for individual multiplications~\cite{handbook}.

Pairing-based cryptosystems on the other hand can benefit from having
all coordinates of points stored in Montgomery representation, with conversion
only during input and output (though even then, there is no reason why keys
and such could not be stored in Montgomery representation). This is because
during point additions and multiplications as well as pairing computations,
there are many finite field operations but never a need to convert the
coordinates back to its normal representation.

Division is slower using Montgomery representation, but this drawback is
of no concern if projective coordinates are used, where each division is
replaced by a few multiplications in any case.

\section{Dedicated Squaring}

For each field, ring or group, dedicated squaring or doubling routines
can be helpful. The savings accrued during exponentiations can be significant.

\section{Field Extensions}

When possible, a degree two field extension of a field $\mathbb{F}_q$
should be implemented as $\mathbb{F}_q[i]$ where $i$ is a square root of $-1$.
This can be done if and only if $q = 3 \pmod{4}$.

We have
\[ (a + i b)^2 = (a - b)(a + b) + i (2 a b) \]
and
\[ (a + i b)(c + i d) = (ac - bd) + i[(a + b)(c + d) - ac - bd] \]

More generally, for low degree extensions (e.g. 3, 6),
hand-coded generalized Karatsuba multiplication algorithms \cite{wpkaratsuba}
will improve running times.

\section{Last-Second Conversions}

One easy optimization is also easy to overlook as it is hidden by
notation.

Consider a type A pairing $e : E(\mathbb{F}_q) \times E(\mathbb{F}_q)
\rightarrow \mathbb{F}_{q^2}$, and suppose we have chosen to
use $\mathbb{F}_q [i]$ to represent $\mathbb{F}_{q^2}$.

During the computation of the pairing,
we evaluate $g(Q')$ where $g = aX + bY + c$ is an equation of some line and
$Q' = \phi(Q)$, where $\phi$ is defined by  $(X,Y) \rightarrow (-X, iY)$
and $Q$ is some point $(x,y)$ in $E(\mathbb{F}_q)$.

If we follow the notation blindly, we would first compute the point
$Q'$ which takes twice as much storage as $Q$ (as the field it lies over is
twice as big), and computing $g(Q')$ involves
operations on elements of $\mathbb{F}_q[i]$.

Instead, it is wiser to never explicitly compute $Q'$, and instead
performs the following:

\begin{enumerate}
\item
$Re(R) = c - a x$
\item
$Im(R) = b y$
\end{enumerate}

Now $R = g(Q')$, and we have only computed in $\mathbb{F}_q$.

Similar statements can be made for other pairing types, with greater
savings in some cases.

\section{Finding a Root of a Polynomial}

When computing parameters for curves using the CM method, we
need to find a root of a Hilbert polynomial modulo a prime.
As we only want a single root, we may use the Cantor-Zassenhaus
method and skip many steps. For example, we do not care about
the multiplicity of the factors or factors of degree higher than 1.
Thus we may do the folliowing to find a root of a
degree $n$ polynomial $f(x)$ in
$\F_q$.

\begin{enumerate}
\item
Compute $g(x) = \gcd(x^q - x, f(x))$.
\item
If $\deg g = 1$ then output the root and stop.
\item
Pick a random $r \in \Fq$. If $r$ is a root then stop.
\item
Compute $s(x) = (x-r)^{(q-1)/2} \bmod g(x)$.
\item
Compute $g'(x) = \gcd(s(x)+1, g(x))$. This is a proper factor of $g$
with probability $1 - 2^{n-1}$. So if $g'(x) \ne 1$ set $g = g'$ and
goto step 2.
\item
Goto step 3.
\end{enumerate}

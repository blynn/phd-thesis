\chapter {Elliptic Curves}
We review some basic facts about elliptic curves.

Let $\Fq$ be a field for some prime $q > 3$.
Unless otherwise specified we shall always
define curves over a field of prime order and of characteristic greater
than three.

An elliptic curve $E$ is an equation of the form
\[ E: Y^2 = X^3 + a X + b \]
where $a, b \in \Fq$.
Let $\Delta = 4 a^3 + 27b^2$ (the discriminant of the cubic in $x$). Then
$E$ is \emph{singular} if $\Delta = 0$ (so the cubic has distinct solutions),
and \emph{nonsingular} otherwise.

Unless otherwise stated we always consider nonsingular elliptic curves.
Note: later we will examine \emph{supersingular} curves which are a
particular kind of nonsingular curves (and are \emph{not} singular).

For every curve, there is a special point $O$, sometimes denoted $\infty$,
called the \emph{point at infinity}
that is always a solution. (By considering the equation in
projective coordinates one can show that $O = (0,1,0)$ is always a unique
infinite solution to the equation. Intuitively,
the point $O$ can be thought of as the point where all lines
parallel to the Y-axis meet.) The other solutions to the equation $E$
are called \emph{finite points}.

For any positive integer $k$ define $E(\Fqk)$ (or $E/\Fqk$)
to be the set of all solutions
of $E$ over $\Fqk$ along with the point $O$. We write $\#E(\Fqk)$ 
or $|E(\Fqk)|$ for
the number of elements of $E(\Fqk)$.

We quote two well-known theorems from the theory of elliptic curves.

\begin{theorem}
[Hasse] Let $t = q^k + 1 - \#E(\Fqk)$.
Then $|t| \le 2\sqrt{q^k}$.
\end{theorem}

Thus the number of points on an elliptic curve is on the same order as the
size of the field it is defined in.

The following theorem is in fact true for any prime power $q$.

\begin{theorem}
[Weil] Let $t = q + 1 - \#E(\Fq)$.
Factor the polynomial $x^2 - t x + q$ as $(x-\alpha)(x-\beta)$ over
$\mathbb{C}[x]$.
Then
\[ \#E(\Fqk) = q^k + 1 - (\alpha^k + \beta^k) .\]
\end{theorem}

This last theorem is more practical in the following form. Let $t_0 = 2$,
Let $t_1 = q + 1 - \#E(\Fq)$.
Define $t_n$ recursively by
\[ t_n = t_1 t_{n-1} - q t_{n-2} .\]
Then $\#E(\Fqk) = q^k + 1 - t_k$.

\section {Twisting Curves}

Let $E$ be an elliptic curve $Y^2 = X^3 + a X + b$ in $\Fq$,
let $v$ be a quadratic nonresidue in $\Fq$. Consider the
curve $E'$ given by $Y^2 = X^3 + a v^2 X + v^3 b$.

Let $t = q + 1 - \#E(\Fq)$.
Then $\#E'(\Fq) = q + 1 + t$.


\section {The Chord-Tangent Composition Law}

We define an operation $+$ on $E(\Fqk)$.

Let $P = (a,b), Q = (c,d) \in E(\Fqk)$ be finite points.

If $P \ne Q$, then it is not hard to show that if $a \ne c$
then the line through $P$ and $Q$ must intersect $E$ at another point
$(x,y)$ where $x, y\in \Fqk$. Note that $(x,-y)$ also is a solution of $E$.
Define $P + Q = (x, -y)$ for $a \ne c$.
If $a = c$ (in which case we must have $b = -d$),
then define $P + Q = O$.

Now suppose $Q = P$. In this case, consider the tangent line going through
$P$. It turns out it must intersect $E$ at another point $(x,y)$ where
$x,y\in\Fqk$ unless $a = 0$. Define $P + P = (x, -y)$ for $a \ne 0$.
For $a = 0$ define $P + P = O$.

Lastly define $P + O = P$, $O + O = O$.

This operation turns $E(\Fqk)$ into a group.
The point $O$ is the identity, and the inverse
of a point $P = (x,y)$ is the point $-P = (x,-y)$.

The previous chapter used mulitplicative group notation to emphasize
the connection
between discrete log cryptosystems and pairing-based cryptosystems.
Traditionally one uses additive group notation for the elliptic curve group
(since the group is Abelian), and we will adhere to that convention.

\section {Torsion Points}

Suppose $P\in E(\Fq)$ has order $r$ (so $r P = O$),
where $r$ is coprime to $q$.

Discrete log cryptosystems can easily be ported from finite fields
to a group $\langle P \rangle$.
Originally, this was the only goal of elliptic curve cryptography,
and researchers focused on finding curves with groups with
suitable orders that were resistant to discrete log attacks and also
efficient to compute on.
In contrast, pairing-based cryptography exploits more facts
about points of a particular order $r$.

It can be proved that for some integer $k \ge 1$,
$E(\Fqk)$ contains $r^2$ points $P$ satisfying $r P = O$,
and there are no other points with this property even if larger field
extensions are considered.

Denote the set of these points by $E[r]$. They are sometimes called the
$r$-torsion points of $E$. It can be shown that

\[ E[r] \cong \mathbb{Z}_r^+ \oplus \mathbb{Z}_r^+ \]

When $r=2$ this is easy to see: a point has order 2 if and only if it has
a zero $y$-coordinate. As $E$ is nonsingular
$x^3 + ax + b = 0$ has three distinct solutions, thus we can always find
some field $\Fqk$ where $E$ has
four points of order 2: the point $O$ and three points of the form
$(\alpha, 0)$ where $\alpha$ is a root of the cubic.

Also, the line through any two of the finite points is simply the line $Y = 0$,
which certainly intersects $E$ the other finite point.
The proof is much less trivial for general $r$.

We shall later construct a function
that takes pairs of elements from $E[r]$ and outputs an $r$th root of unity
that is bilinear and nondegnerate.

\section {Representing Points}

Projective coords. Mixin.

optim for Space vs. time

\section {Point Addition}

\section {Point Multiplication}

Preprocessing. Sliding Windows.

\section {Finding Random Points}

\section {Rational Functions}

Consider the field $\Fqk(X,Y)_{E(X,Y)}$.

\section {Curve Endomorphisms}

Given a point $P$ and integer $m$ consider the \emph{multiplication by $m$ map}
$[m]$ given by \[ P \mapsto m P . \]
If $P$ is viewed as a pair of variables $(X,Y)$, this map
can be written as $(f(X,Y), g(X,Y))$ for some rational functions $f,g$.

This is an example of a \emph{curve endomorphism}, because points of
$E(\Fqk)$ are mapped to points of $E(\Fqk)$. Another important example
is the \emph{Frobenius map} $\Phi$, given by $(X, Y) \mapsto (X^q, Y^q)$.

\begin{theorem}
[Hasse]
\[ \Phi \circ \Phi - [t] \circ \Phi + [q] = [0] \]
\end{theorem}

Since $E[r] \cong \mathbb{Z}_r \times \mathbb{Z}_r$ any curve endomorphism
can be viewed as a $2\times2$ matrix when restricted to $E[r]$.

TODO: eigenvalues of Phi

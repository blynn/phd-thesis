\chapter {Elliptic Curves}

We review basic facts about elliptic curves.
Aside from the problem of generating suitable elliptic curves
and counting the size of the resulting group,
we will cover enough theory to replace finite fields with
elliptic curves in cryptosystems based on cyclic groups.

We discuss pairings and algorithms to find pairing-friendly curves
in future chapters. We will not
discuss curve-finding and point-counting algorithms geared
towards standard elliptic curve cryptography which necessarily requires
curves that are not pairing-friendly, but instead direct
the interested reader to Blake, Seroussi and Smart~\cite{bss}.

\section {Informal Overview of Elliptic Curves}

We present a highly informal overview to elliptic curve theory,
which may aid those that have not encountered them before.
This section can be safely skipped.

Consider a polynomial $C$ in two variables $X, Y$.
We are interested in the solutions to $C = 0$
which describe a curve on a two-dimensional plane.

We first observe that if $C'$ is another curve that is an affine transformation
of $C$, that is, if we can linearly transform (e.g. rotate, scale, shear)
and then translate $C$ to
obtain $C'$ then a correspondence exists between
the solutions to $C = 0$ and the solutions to $C' = 0$. Knowing
the solutions of one allows us to easily compute the solutions of the other,
For this reason we consider such curves $C$ and $C'$ to be equivalent.

If every term in $C$ has combined degree of at most 1, that is,
if $C = a X + b Y + c$ then $C$ describes a line. The geometry of lines
is too simple to yield anything cryptographically useful.

If every term in $C$ has combined degree at most 2, then $C$ describes
a single line, a pair of lines, an ellipse, a parabola or a hyperbola.
The first two possiblities can be viewed as special cases, occuring
when $C$ is reducible or degenerate in some sense.

By adding \emph{points of infinity} to the plane, we can find affine
transformations that change any ellipse, parabola or hyperbola into the
unit circle centred at the origin. Intuitively, the two ends of the parabola
can be thought of as meeting
at a point at infinity, forming a circle,
and similarly opposite ends of hyperbolas connect at infinite points.

Thus to study degree 2 curves is essentially to study the unit circle,
whose geometry is again is too simple for our purposes.

However, degree 3 curves, called elliptic curves,
are nontrivial (for instance, unlike the previous two
cases we cannot transform any elliptic curve into
any other)
and have a rich structure well-suited for cryptography.

For any irreducible elliptic curve $C$,
applying appropriate affine transformations
produces an equation $C'$ of a certain form known
as the \emph{Weierstrass form}.
In this form, $C'$ always contains exactly one point at infinity.
Transforming $C$ so
that no points are infinite is
possible but leads to more complicated
equations. Using an equation that has
exactly one point at infinity simplifies some of the expressions.

\section {Points on Elliptic Curves}

Let $\Fq$ be a field for some prime $q > 3$.
Unless otherwise specified we shall always
define curves over a field of prime order and of characteristic greater
than three.
Elliptic curves can be implemented over fields of characteristic 2 and 3
and enjoy many optimizations,
but suffer from specialized discrete log attacks and should generally
be avoided.

An elliptic curve $E$ over such a field $\Fq$ is an equation of the form
\[ E: Y^2 = X^3 + a X + b \]
where $a, b \in \Fq$.
Let $\Delta = 4 a^3 + 27b^2$, the discriminant of the cubic in $x$. Then
$E$ is \emph{singular} if $\Delta = 0$, i.e. the cubic has a repeated root,
and \emph{nonsingular} otherwise, i.e. the cubic has distinct roots.

Unless otherwise stated we always consider nonsingular elliptic curves.
Later we will encounter \emph{supersingular} curves, a particular
breed of nonsingular curves and are not singular despite their name.
The antonym of supersingular is \emph{ordinary} or \emph{nonsupersingular}.

For any field $\Fqk$ define $E(\Fqk)$ (or $E/\Fqk$)
to be the set of all solutions
of $E$ over $\Fqk$, called the \emph{finite points}
along with a special point denoted $O$,
that is called the \emph{point at infinity}. We write $\#E(\Fqk)$ 
or $|E(\Fqk)|$ for
the number of elements of $E(\Fqk)$.

Although irrelevant, the reader may be curious about the point at infinity.
Briefly, by considering the equation in
projective coordinates one can show that $O = (0,1,0)$ is always a unique
infinite solution to the equation. Intuitively,
the point $O$ can be thought of as the point where all lines
parallel to the Y-axis meet.

We quote two well-known theorems.

\begin{theorem}
[Hasse] Let $t = q^k + 1 - \#E(\Fqk)$.
Then $|t| \le 2\sqrt{q^k}$.
\end{theorem}

Thus the number of points on an elliptic curve in a given field
is on the same order as the size of the field.
The quantity $t$ is called the \emph{trace of Frobenius}.

\begin{theorem}
[Weil] Let $t = q + 1 - \#E(\Fq)$
where $q$ is a prime power.
Factor the polynomial $x^2 - t x + q$ as $(x-\alpha)(x-\beta)$ over
$\mathbb{C}[x]$.
Then
\[ \#E(\Fqk) = q^k + 1 - (\alpha^k + \beta^k) .\]
\end{theorem}

This last theorem is more practical in the following form. Let $t_0 = 2$,
Let $t_1 = q + 1 - \#E(\Fq)$.
Define $t_n$ recursively by
\[ t_n = t_1 t_{n-1} - q t_{n-2} .\]
Then $\#E(\Fqk) = q^k + 1 - t_k$.

For example, consider the curve E\cite{bk} given by
\[ Y^2 = X^3 + X + 6 \]
over $\F_{19}$. There are 18 points
\[
\begin{array}{llllll}
(0,5), & (4,6), & (2,4), & (3,6), & (14,3), & (12,13), \\
(18,2), & (10,3), & (6,0), & (10,16), & (18, 17), & (12,6), \\
(14,16), & (3,13), & (2,15), & (4,13), & (0,14), & O
\end{array}
\]
thus the trace of Frobenius $t = 2$.

Over $\F_{{19}^2}$, we have
\[
\#E(\F_{{19}^2}) = 19^2 + 1 - t_2
\]
where $t_2 = 2\cdot 2 - 19 \cdot 2 = -34$, thus
$\#E(\F_{{19}^2}) = 396$.

\section {Finding Points}

Let $E : Y^2 = X^3 + a X + b$
be an elliptic curve over a field $K$.
There always exists an unique infinite solution, namely $O$.
We describe a simple method for finding the finite points of $E$.

For any $x \in K$, we may attempt to
solve $Y^2 = x^3 + a x + b$ for $Y$ by finding a square root
of the right-hand side. We momentarily postpone describing the
details of square root algorithms.
For now, assume we can find square roots.

When solutions for $Y$ do exist for a given $x$,
we have found exactly two points, one for each square root, except
in the rare case when the point lies on the $X$-axis, which can happen in
at most three places.

Also, recall from an above theorem that the size of $K$ is roughly the same
as the number of points on $E(K)$.

Combining these two facts shows that
for approximately half of the choices for $x \in K$,
a square root exists and we can solve $E$ to
find a point. Thus we have a fast method of finding random points on
$E$:

\begin{enumerate}
\item
Choose $x \in K$ at random.
\item
Solve $Y^2 = x^3 + a x + b$ for $Y$. If there are no solutions
then go to the previous step.
\item
Flip a coin to decide which solution of $Y$ to use.
\end{enumerate}

Of course, it is impossible to choose the point at infinity with this method,
and points that lie on the $X$-axis have a slightly higher probability of
been picked than other points. For cryptography this is of no concern
since the point of infinity is usually unwanted, and the probability
of finishing at a point with zero $Y$-coordinate is negligible since there
are at most three of them. Moreover, it is often unimportant which
square root is chosen.

If one insists on choosing all points of $E(K)$ uniformly, one could simply
add a step before choosing $x$.
Let $n = \#E(K)$. Then with
with $1/n$ probability,
choose $O$ or one of the points lying
on the $X$-axis, otherwise proceed with the above algorithm, except
in the second step, we also go back to the first step if the only solution
is $Y = 0$.

Before attempting to find a square root of a given element $x \in K$,
we can check that one actually exists first.
When $K$ has prime order,
one can compute the Legendre symbol before attempting to square root $x$.
More generally it can be checked that $X^2 - x$ is reducible.

Alternatively,
one can omit the check, proceed with a square root algorithm,
and compare the square of the output with $x$: if there is a mismatch
then $x$ is not a square after all.

It remains to describe how to take square roots.
For a field of prime order one can use the Tonelli-Shanks algorithm
to compute square roots~\cite{bss, handbook}, which we quote below.

For a general finite field,
one must use a more complex algorithm. Perhaps the simplest of these
Legendre's method which can be viewed as a special
case of the Cantor-Zassenhaus algorithm for factoring polynomials that
will be described in a later chapter.
Faster algorithms exist, though sometimes require precomputation~\cite{djb}.

\subsection {Tonelli-Shanks Algorithm}

Suppose we wish to compute $b = \sqrt{a}$ in
a field $\Fq$ for some prime $q$. (These paragraphs should be viewed
as self-contained; the notation from earlier paragraphs does not apply here.)

\begin{algorithm}
\caption{(Tonelli-Shanks) Find $b =\sqrt{a}$ in a prime field}
\begin{algorithmic}[1]
\STATE Find an element $g \in \Fq$ that is not a square.
\STATE Since $q$ is odd (unless $q = 2$ in which case square roots are trivial),
we may write $q - 1 = 2^s t$ for some odd $t$.
\STATE $e \gets 0$
\FOR {$i \gets 2$ to $s$}
\IF {$(ag^{-e})^{(q-1)/2^i} \ne 1$}
\STATE $e \gets 2^{i-1} + e$
\ENDIF
\ENDFOR
\STATE $h \gets ag^{-e}$
\STATE $b \gets g^{e/2} h^{(t+1)/2}$
\STATE Return $b$
\end{algorithmic}
\end{algorithm}

The first step can be accomplished
by choosing random elements $g \in\Fq$ until $g^{(q-1)/2} = -1$. 
Clearly this $g$ can be stored for use in future square root computations
in the same field.

We can briefly explain the Tonelli-Shanks algorithm as follows.
Observe square roots in a cyclic group of order $t$ where $t$ is odd can be
computed by exponentiating by $(t+1)/2$.
Then using $\Fq^* \cong \mathbb{Z}_{2^s}^+
\times \mathbb{Z}_t^+$ for some odd $t$ leads to the above.

\subsection {Hashing to Points}

Finding points by choosing an $X$-coordinate and solving for $Y$
suggests an efficient algorithm for hashing to a point in $E$.
The input is hashed to some $x \in K$, and then a corresponding $y$ is sought.
On failure, a new $x$-coordinate is deterministically generated from $x$,
and again we attempt to solve $E$ for $y$. Repeating this process as many
times as
necessary eventually yields a valid point $(x,y) \in E(K)$.

\section {Point Compression and Reduction}

Another implication of taking square roots to find a $Y$ corresponding to
a particular $X$ value is that
a point $(x,y)$ can be represented by $x$ along with a single bit indicating
which solution of $y$ to take. This technique is known as \emph{point
compression}.

Alternatively, for some cases we can simply use $x$ alone, which
is sometimes called \emph{point reduction}. For example,
suppose Alice needs to send Bob a point in some cryptosystem. She sends
Bob $x$, who guesses the solution $y$. Bob attempts to proceed normally.
If the protocol fails e.g. the signature does not verify, then Bob tries again
with $-y$. This does not cost much more since the solutions are
related by negation, and in Section~\ref{sec:pairingcompressioneven}
we show how to check both possibilities with only one operation.

Trivially we can take this principle further. Alice can
omit $k$ bits of $x$, and leave Bob to try all $2^k$ possibilities.

\section {The Chord-Tangent Law of Composition}

We define an operation $+$ on $E(\Fqk)$.
Let $P = (a,b), Q = (c,d) \in E(\Fqk)$ be finite points.

If $P \ne Q$, then it is not hard to show that if $a \ne c$
then the line through $P$ and $Q$ must intersect $E$ at another point
$(x,y)$ where $x, y\in \Fqk$. Note that $(x,-y)$ also is a solution of $E$.
Define $P + Q = (x, -y)$ for $a \ne c$.
If $a = c$ (in which case we must have $b = -d$),
then define $P + Q = O$.

Now suppose $Q = P$. In this case, consider the tangent line going through
$P$. It turns out it must intersect $E$ at another point $(x,y)$ where
$x,y\in\Fqk$ unless $b = 0$. Define $P + P = (x, -y)$ for $b \ne 0$.
For $b = 0$ define $P + P = O$.

Lastly define $P + O = P$, $O + O = O$.

This operation turns $E(\Fqk)$ into a group.
The point $O$ is the identity, and the inverse
of a point $P = (x,y)$ is the point $-P = (x,-y)$.

As usual, define $0 P = O$, $1 P = P$,
$n P = (n-1)P + P$ for integers $n > 1$ and $n P = -(-n)P$ for integers
$n < 0$.
This operation is termed \emph{point multiplication}.
Point multiplication can be performed efficiently via carefully
chosen point additions, in a process that mirrors the repeated
squaring technique for exponentiation in finite fields.

Recall the previous chapter used mulitplicative group notation to emphasize
the connection
between discrete log cryptosystems and pairing-based cryptosystems:
we are using elliptic curve groups where one would use
the multiplicative group of a finite field.

Mathematicians use additive group notation for the elliptic curve group
since the group is Abelian, and we will adhere to this convention in this
section.

\section {Explicit Formulas}

Let $E: y^2 = x^3 + ax + b$ be an elliptic curve over $\Fqk$. Let
$P_1, P_2 \in E(\Fqk)$ and suppose
we wish to find $P_3 = P_1 + P_2$.
Note the case where at least one of $P_1, P_2$ is $O$ is trivial,
so assume both $P_1$ and $P_2$ are finite.
Then write $P_1 = (x_1, y_1), P_2 = (x_2, y_2)$.

If $x_1 = x_2$ and $y_1 = -y_2$
then the line $V$ through $P_1$ and $P_2$ is vertical and
can be given by
\[ V : X - x_1 = 0  . \]
In this case $P_3 = O$ (and $P_1 = -P_2$).

Otherwise $P_3$ also must be finite, hence write $P_3 = (x_3, y_3)$.
We have two cases. If $P_1 = P_2$ then the tangent line $T$ at
$P_1$
has slope
\[\lambda =
\left[ \frac{\partial E / \partial X}{\partial E/ \partial Y} \right]_{(x_1,y_1)}
= \frac{3x_1^2 + a}{2y_1} . \]
and if $P_1 \ne P_2$ then the line $L$ through $P_1$ and $P_2$ has slope
\[ \lambda = (y_2 - y_1)/(x_2 - x_1) . \]

If we set $\mu = y_1 - \lambda x_1$ then
the equation of $L$ or $T$ is given by:

\[ L, T :  Y - (\lambda X + \mu) = 0 \]

We can find the $x$-coordinate of the other point of intersection $(x_3, -y_3)$
by substituting $Y = \lambda X + \mu$ into $E$. We find
\[ E(X, \lambda X+\mu) = X^3 - \lambda^2 X^2 + a_1 X + a_0 = 0 \]
for some $a_1, a_0 \in \Fqk$, thus the sum of the roots
$x_1 + x_2 + x_3 = \lambda^2$.

This allows us to compute $x_3, y_3$. Explicitly we have for $P_1 \ne P_2$:

\[
\begin{array}{rcl}
\lambda &\gets& (y_2 - y_1)/(x_2 - x_1) \\
x_3 &\gets& \lambda^2 - x_1 - x_2 \\
y_3 &\gets& (x_1 - x_3) \lambda - y_1
\end{array}
\]
and for $P_1 = P_2$:
\[
\begin{array}{rcl}
\lambda &\gets& (3x_1^2 + a)/(2y_1) \\
x_3 &\gets& \lambda^2 - 2x_1 \\
y_3 &\gets& (x_1 - x_3) \lambda - y_1
\end{array}
\]
The most expensive step is the division in the computation of
$\lambda$.

The expressions for the lines $L,T,V$ will be used later in the computation
of a pairing.

For the curve $Y^2 = X^3 + X + 6$ over $\F_{19}$,
using these formulas, it can be verified for instance that
$(0, 5) + (0, 5) = (4, 6)$,
and $(0, 5) + (2, 15) = (4, 13)$.

\section {Elliptic Curve Cryptography}

We now possess enough theory to show how elliptic curves may be used
in cryptography.
Let $E$ be an elliptic curve over a field $K$.
The group operation described above means that every point on $E$
generates a cyclic group $G$.
Then we can use $G$ for cyclic group cryptography
provided that its order is prime,
that the basic operations, namely group operation, inversion, hashing,
are efficient, and that problems such as discrete log are difficult.

The formulas above show that only a few operations in $K$ are required for
point addition and negation.
We previously saw how to hash to points in $E(K)$.
Thus it seems elliptic curve cryptography can replace
cryptography in finite fields by using
points in $E(K)$ instead of elements of some $F^*$ for some finite field $F$,
and the group operation is point addition instead of modular multiplication.
The only obstacle is ensuring that randomly chosen points and hashed points
lie in $G$, and not all of $E(K)$.

Recall that cryptographic schemes in $F^*$ for some finite field $F$
often operate within a subgroup $G$ of a particular order $r$,
so elements chosen at random and hashed to must have
order $r$, or a factor of $r$. But an element of $F^*$
in general does not have such an order.
Thus after using some algorithm to
choose or hash to some element $x \in F^*$,
to obtain an element of a suitable order one simply exponentiates $x$
by $n / r$ where $n = \#F^*$.
On elliptic curves, the construction of a point of order $r$,
or a factor of $r$, from some given point $P \in E(K)$ can be accomplished
similarly by multiplying $P$ by
$n / r$ where $n = \#E(K)$.

Let $n = \#E(K)$. Then from Abelian group theory
for any prime $r$ dividing $n$,
there exists a point $P \in E(K)$ of order $r$.
and furthermore, if $r^2$ does not divide $n$ then there is exactly one
subgroup $G$ of $E(K)$ of order $r$.

This suggests the following procedure for implementing any cryptographic
scheme based on cyclic groups of prime order:

\begin{enumerate}
\item
Choose any curve $E(K)$ and somehow work out $n = \#E(K)$.
\item
Find a prime $r$ dividing $n$, such that $r^2 \nmid n$.
We shall work in the unique cyclic subgroup $G \subset E(K)$ of points
of order $r$.
\item
When a random group element of $G$ is required, first choose a random point
of $E(K)$ and then multiply by $n / r$. Similarly, when hashing to
a point of $G$, first hash to a point in $E(K)$ and then mulitply by
$n / r$.
\item
Other operations are straightforward: every time a group
operation is required, we perform a point addition. To find an inverse
of a group element, we negate the $y$-coordinate of a point. When
an exponentiation is called for we carry out a point multiplication
\end{enumerate}

This easily generalizes to squarefree $r$.

In some applications it does not matter if the group is not
cylic. For example, many cryptosystems function equally well
in a group isomorphic to $\mathbb{Z}_{r}^+ \times \mathbb{Z}_{r}^+$.
In these cases we allow $r^2$ to divide $n$.

We quote another well-known theorem which implies that we
never have more than two copies of $\mathbb{Z}_r^+$ in
$E(K)$.
Note our most general abstract definition of the pairing in fact permits
groups with at least three copies of a cyclic group, even though this
theorem shows this is impossible for elliptic curves.

\begin{theorem}
\[ E(K) = \mathbb{Z}_r^+ \times \mathbb{Z}_s^+ \]
for some integers $r,s$ with $r \mid s$.
\end{theorem}

In the running example, $E : Y^2 = X^3 + X + 6$ over $\F_{19}$,
the order of the group is $n = 18 = 3^2 \times 2$, and it so happens
that the group is cyclic: it can be checked that $(0,5)$ is a generator.
Thus this curve can be used
for cryptography on a cyclic group of order $r = 3$. However, in general,
if $r^2 | n$ there may be more than one subgroup of order $r$ and the
above procedure cannot be followed.

We have yet to describe how to compute $\#E(K)$ for a given curve.
Fast algorithms exist for this~\cite{bss}
but it turns out that if a pairing is desired, we must seek out
elliptic curves whose orders satisfy various conditions.
As a result,
instead of choosing a curve first and counting the number of points $n$ and
hoping for a large prime factor $r$ of $n$,
we must use families of curves where the size of the group
is known in advance and has the requisite properties,
a subject of a later chapter.

For now, we exhibit one case where $\#E(K)$ is always easy to determine
and furthermore $E(K)$ is always cyclic.

\section {Singular Elliptic Curves}

Using singular elliptic curves
is equivalent to conventional cryptography.
It can be shown that the set $E_{ns}$ nonsingular points (all but
one) of a singular elliptic curve over a field $K$
also form a group under the chord-tangent law,
and $E_{ns} \cong K^*$ or $E_{ns} \cong K^+$ \cite[Proposition 2.5]{silverman}.
Furthermore these isomorphisms are efficiently computable in either
direction.
Note the latter case is useless for cryptographically
as discrete log is easy in $K^+$.

For example, consider the curve
\[ E : y^2 = x^3 + x^2 \]
This has a singular point at $(0,0)$. Then the other solutions to
this curve (including the point at infinity) form a group
$E_{ns}(\Fqk)$ for any finite field $\Fqk$.

Furthermore, there is an isomorphism $E_{ns}(\Fqk) \rightarrow \Fqk^*$
given by
\[ (x,y) \mapsto \frac{y-x}{y+x} \]
(and maps $O$ to $1$)
which is efficienty computable in either direction.
A little algebra
shows the reverse map is
\[ z \mapsto \left( x', \frac{1+z}{1-z} x' \right) \]
where $x' = 4z/(1-z)^2$ for $z \ne 1$, and $1 \mapsto O$.

This example has little practical value as point additions are slower than
modular multiplications, and keeping both coordinates of a point in memory
takes twice as much space. However, it does suggest
that cryptography in finite fields can be viewed as a special
case of elliptic curve cryptography, and strictly speaking,
those that complain that elliptic curve cryptography is controversial,
too experimental and untested must qualify
their remarks by making exceptions for singular curves!

It is clear that if an efficiently bilinear nondegenerate pairing were found
for singular curves then the Decisional Diffie-Hellman problem could be broken
in finite fields.

\section {\label{sec:eccsec}Security}

We have not yet shown any benefits to using elliptic curves instead
of finite fields. On the contrary, although a group inversion is
almost free, the group operation, hashing, and random element generation
are considerably more expensive for the same field.

Their strength derives from the fact that no discrete log algorithm for
general elliptic curves that outperforms generic methods has ever been
discovered.

The best generic methods are based on the birthday paradox, and have
$O(\sqrt{n})$ expected running time, where $n$ is the group order.
A group order around 160 bits in length is sufficient to defeat such attacks.
In contrast, subexponential discrete log algorithms for finite fields
mean that one must use at least 1024-bit finite fields for security.

Hence one may work with fields over six times smaller in length
if elliptic curves are used. The increase in speed in
the underlying field arithmetic easily compensates for the more
complex group operations. Thus choosing elliptic curves instead of
finite fields results in savings in both space and time.

The most important attribute of elliptic curves for our purposes
is that without them, we cannot construct a cryptographically
useful bilinear map. That elliptic curves happen to be more
efficient than finite fields
is a happy coincidence and makes pairings even more attractive.

\section {Short Signatures}

At this point we can see why
the BLS signature scheme introduced
in the first chapter features a short signature length.
Recall in abstract terms,
a signature is a single element of a cyclic group $G$ of prime order $r$
and we need a bilinear map to exist on $G$.

In Section~\ref{sec:comparingpairings} we give examples
of pairings where one input group is a subgroup of points
on an elliptic curve $E$ over some field $\Fq$ where
$q$ is about 160 bits long.

A point has two coordinates, both taking about 160 bits to represent.
We can use point reduction, that is, discard the $y$-coordinate entirely,
and represent signatures by their $x$-coordinate only if we modify
verification as follows:
\begin{enumerate}
\item
Given an $x$-coordinate of a signature, find any solution $y$
in the curve equation $E$.
\item
If the signature $(x,y)$ does not verify,
check if the signature $(x, -y)$ verifies.
\item
If the signature still does not verify then reject it.
\end{enumerate}

Thus signatures are elements of $\Fq$ and hence roughly 160 bits in length.

Verification of the signature involves computing $e(P, Q)$ for some points
$P, Q$. By the bilinearity of the pairing, if we have guessed the point $Q$
wrongly, we can obtain the value of $e(P,-Q) = e(P,Q)^{-1}$ by performing
an inversion rather than recompute another pairing.

In Section~\ref{sec:compressedpairings} we will see we can do better
than this by using a technique known
as \emph{pairing compression}, and in effect check both
cases at once.

Instead of discarding the $y$-coordinate, we can
replace it with a single bit signifying which solution of $y$ to take,
a technique known as \emph{point compression}~\cite[Section IV.4]{bss}.

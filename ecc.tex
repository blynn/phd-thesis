\chapter {Elliptic Curves}
We review some basic facts about elliptic curves.

Let $\Fq$ be a field for some prime $q > 3$.
Unless otherwise specified we shall always
define curves over a field of prime order and of characteristic greater
than three.

An elliptic curve $E$ is an equation of the form
\[ E: Y^2 = X^3 + a X + b \]
where $a, b \in \Fq$.

For every curve, there is a unique point $O$,
called the \emph{point at infinity}
that is always a solution. (By considering the equation in
projective coordinates one can show that $O = (0,1,0)$ is the unique
infinite solution to the equation. On the real plane,
the point $O$ can intuitively thought of as the point where all lines
parallel to the Y-axis meet.) The other solutions to the equation $E$
are called \emph{finite points}.

For any positive integer $k$ define $E(\Fqk)$ to be the set of all solutions
of $E$ over $\Fqk$ along with the point $O$.

\section {The Chord-Tangent Composition Law}

We define an operation $+$ on $E(\Fqk)$.

Let $P = (a,b), Q = (c,d) \in E(\Fqk)$ be finite points.

If $P \ne Q$, then it is not hard to show that if $a \ne c$
then the line through $P$ and $Q$ must intersect $E$ at another point
$(x,y)$ where $x, y\in \Fqk$. Note that $(x,-y)$ also is a solution of $E$.
Define $P + Q = (x, -y)$ for $a \ne c$.
If $a = c$ (in which case we must have $b = -d$),
then define $P + Q = O$.

Now suppose $Q = P$. In this case, consider the tangent line going through
$P$. It turns out it must intersect $E$ at another point $(x,y)$ where
$x,y\in\Fqk$ unless $a = 0$. Define $P + P = (x, -y)$ for $a \ne 0$.
For $a = 0$ define $P + P = O$.

Lastly define $P + O = P$, $O + O = O$.

This operation turns $E(\Fqk)$ into a group.
The point $O$ is the identity, and the inverse
of a point $P = (x,y)$ is the point $-P = (x,-y)$.

The previous chapter used mulitplicative group notation to emphasize
the connection
between discrete log cryptosystems and pairing-based cryptosystems.
Traditionally one uses additive group notation for the elliptic curve group
(since the group is Abelian), and we will adhere to that convention.

\section {Torsion Points}

Let $P\in E(\Fq)$ have order $r$ (so $r P = O$),
where $r$ is coprime to $q$.

Discrete log cryptosystems can easily be ported from finite fields
to such a group $\langle P \rangle$.
Originally, this was the only goal of elliptic curve cryptography,
and researchers focused on finding curves with groups with
suitable orders that were resistant to discrete log attacks and also
efficient to compute on.
On the other hand, pairing-based cryptography exploits more facts
about points of a given order.

It can be proved that for some integer $k \ge 1$,
$E(\Fqk)$ contains $r^2$ points $P$ satisfying $r P = O$,
and there are no other points with this property even if larger field
extensions are considered.

Denote the set of these points by $E[r]$. They are sometimes called the
$r$-torsion points of $E$. It can be shown that

\[ E[r] \cong \mathbb{Z}_r^+ \oplus \mathbb{Z}_r^+ \]

We shall later describe a function known as the Weil pairing
that takes pairs of elements from $E[r]$ and outputs an $r$th root of unity
that can be used to construct bilinear maps.

\section {Representing Points}

Projective coords. Mixin.

optim for Space vs. time

\section {Point Addition}

\section {Point Multiplication}

Preprocessing. Sliding Windows.

\section {Finding Random Points}

\section {Rational Functions}

Consider the field $\Fqk(X,Y)_{E(X,Y)}$.

\section {Curve Endomorphisms}

Given a point $P$ and integer $m$ consider the \emph{multiplication by $m$ map}
\[ P \mapsto m P . \]
If $P$ is viewed as a pair of variables $(X,Y)$, this map
can be written as $(f(X,Y), g(X,Y))$ for some rational functions $f,g$.

This is an example of a \emph{curve endomorphism}, because points of
$E(\Fqk)$ are mapped to points of $E(\Fqk)$. Another important example
is the \emph{Frobenius map}, given by $(X, Y) \mapsto (X^q, Y^q)$.

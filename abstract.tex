
We define bilinear maps, or pairings, and show how they
give rise to cryptosystems with new functionality.

There is only one known mathematical setting where desirable
pairings exist: hyperelliptic curves. We focus on elliptic curves,
which are the simplest case, and also the only curves used in practice.
All existing implementations of pairing-based cryptosystems are built
with elliptic curves. Accordingly,
we provide a brief overview of elliptic curves, and functions known as
the Tate and Weil pairings from which cryptographic pairings are derived.

We describe several methods for obtaining curves that yield Tate and Weil
pairings that are efficiently computable yet are still cryptographically
secure.

We discuss many optimizations that greatly reduce the running time
of a naive implementation of any pairing-based cryptosystem. These
techniques were used to reduce the
cost of a pairing from several minutes
to several milliseconds on a modern consumer-level machine.

Applications of pairings are largely beyond our scope, but we
do show how pairings allow the construction of a digital signature scheme
with the shortest known signature lengths at typical security levels.
